\documentclass[a4paper, english, 11pt]{article}

\usepackage{setspace}
\onehalfspacing
\usepackage[T1]{fontenc}
\usepackage{hyperref}
\usepackage[utf8]{inputenc}
\usepackage{dcolumn}
\usepackage{color}
\usepackage[x11names]{xcolor}
\usepackage[thmmarks]{ntheorem}
\usepackage{ragged2e}
\usepackage{pdfpages}
\usepackage{amsmath}
\usepackage[french, english]{babel}
\usepackage{multirow}
\usepackage{array}
\usepackage{booktabs}
\usepackage{wrapfig}
\usepackage{multicol}
\usepackage{algorithm}
\usepackage{lscape}
\usepackage{pifont}
\usepackage{graphicx}
\usepackage{cite}
\usepackage{calc}

\usepackage[all]{xy}
\usepackage{framed}
\usepackage{tikz}
\def\dar[#1#2]{\ar@<2pt>[#1]\ar@<-2pt>[#2]}
\newcommand*\circled[1]{\tikz[baseline=(char.base)]{
            \node[shape=circle,draw,inner sep=2pt] (char) {#1};}}
\usepackage{amsfonts,amssymb,amsmath}
\usepackage[top=3cm,bottom=3cm,left=3cm,right=3cm]{geometry}
\usepackage{fancyhdr}
%\pagestyle{fancy}
\usepackage{array}
\newcommand{\fonction}[5]{\begin{array}{lccc}
#1: & #2 & \longrightarrow & #3 \\
    & #4 & \longmapsto & #5 \end{array}}
\newcommand{\Si}{\emph{Si}}
\newcommand{\Alors}{\emph{Alors}}
\newcommand{\hyp}[3]{\Si : \begin{tabular}{lll}
$\star$ #1\\
$\star$ #2\\
$\star$ #3\\
\end{tabular}}
\newcommand{\hypo}[2]{\Si : \begin{tabular}{ll}
$\star$ #1\\
$\star$ #2\\
\end{tabular}}
\newcommand{\grad}{\vec{\nabla}}
\newcommand{\rot}[1]{\vec{\nabla}\wedge\vec{#1}}
\newcommand{\dive}[1]{\vec{\nabla}\cdot\vec{#1}}
\newcommand{\lap}{\vec{\nabla}^2}
\newcommand{\derp}[3]{\displaystyle{\bigg(\frac{\partial #1}{\partial #2}\bigg)_{#3}}}
\newcommand{\imp}[1]{\emph{\textbf{#1}}}
\newcommand{\K}[0]{\mathbb{K}}
\newcommand{\kk}[0]{\textbf{k}}
\newcommand{\Mod}[0]{\text{Mod}}
\newcommand{\A}{$A_\infty$}
\newcommand{\Bb}{\vec{B}}
\newcommand{\E}{\vec{E}}
\newcommand{\Pe}{\text{Pers}}
\newcommand{\0}{\vec{0}}
\newcommand{\R}[0]{\mathbb{R}}
\newcommand{\N}[0]{\mathbb{N}}
\newcommand{\Q}[0]{\mathbb{Q}}
\newcommand{\Z}[0]{\mathbb{Z}}
\newcommand{\C}[0]{\mathcal{C}}
\newcommand{\Orth}[0]{\mathcal{O}}
\newcommand{\Sym}[0]{\mathcal{S}}
\newcommand{\GL}[0]{\mathcal{GL}}
\newcommand{\M}[0]{\mathfrak{M}}
\newcommand{\B}[0]{\mathcal{B}}
\newcommand{\F}[0]{\mathcal{F}}
\newcommand{\G}[0]{\text{R}\Gamma}
\newcommand{\Ll}[0]{\mathcal{L}}
\newcommand{\U}[0]{\mathcal{U}}
\newcommand{\V}[0]{\mathbb{V}}
\newcommand{\D}[0]{\text{D}}
\newcommand{\Obj}[0]{\text{Obj}}
\newcommand{\Ho}[0]{\text{H}}
\newcommand{\Op}[0]{\text{Op}}
\newcommand{\Ouv}[0]{\mathrm{Open}}
\newcommand{\op}[0]{\text{op}}
\newcommand{\Hom}[0]{\text{Hom}}
\newcommand{\Pers}[0]{\text{Pers}}
\newcommand{\Rr}[0]{\text{R}}
\newcommand{\MV}{\text{MV}}
\newcommand{\s}{\textbf{sf}}

\newcounter{nfigure} 
\makeatletter
 
\newskip\@bigflushglue \@bigflushglue = -100pt plus 1fil
 
\def\bigcenter{\trivlist \bigcentering\item\relax}
\def\bigcentering{\let\\\@centercr\rightskip\@bigflushglue%
\leftskip\@bigflushglue
\parindent\z@\parfillskip\z@skip}
\def\endbigcenter{\endtrivlist}
 
%\bibliographystyle{plain}
%\bibliography{biblio_quivers.bib}

\setcounter{nfigure}{0} 
\newcommand{\nfigure}{ \stepcounter{nfigure} \textsc{\textbf{ Figure  \arabic{nfigure}}} - }



\theoremheaderfont{\scshape\bfseries}
\theorembodyfont{\normalfont}
\newtheorem{prop}{Proposition}[section] 

{ \theoremsymbol{\flushright{$\square$}} 
 \newtheorem*{pf}{Proof} }
 
  \newtheorem{rem}[prop]{Remarque} 
 \newtheorem{ex}[prop]{Example} 
 \newtheorem{cor}[prop]{Corollary}
 \newtheorem{remark}[prop]{Remark}
 \newtheorem{lem}[prop]{Lemma}
 \newtheorem{defi}[prop]{Definition}
 \newtheorem{thm}[prop]{Theorem}
\author{Nicolas Berkouk, Grégory Ginot, Steve Oudot}





\begin{document}
\selectlanguage{english}

\title{Level set persistence and sheaf theory}
\maketitle
\begin{abstract}
   
\end{abstract}

\tableofcontents

\section{Introduction}

\subsection{Notations}

\begin{itemize}
\item We fix a ground field $\kk$.
    \item For $s=(s_1,s_2)\in \R^2_{>0}$, we will use the notations $s_x:=(s_1,0)$ and $s_y := (0,s_2)$
    \item Given a category $\C$, we denote $\C^{\op}$ its opposite category. 
    \item We will use the same notation for a poset $(S, \leq)$ and its associated  category whose objects are the elements of $S$ and whose set of morphims from  $s$ to $t$ consists of a single element if $s\leq t$ and is empty if not.
    \item A functor $M$ from a post $(S, \leq)$ to vector spaces is called \emph{pointwise finite dimensional}, \emph{pfd} for short, if for every $s\in S$, $M(s)$ is finite dimensional. 
    \item For a death block $B$, define its dual $B^\dag$ in $\R^2$, and vice-versa.
    \item We denote $t\mapsto \vec{t}$ the functor $(\R ,\leq) \to (\Delta^+, \leq)$.
    \item Preciser la notation $M^h$ dans la proposition de stabilite 2.21 et que ca donne des exemples canoniques de MV.
\end{itemize}

unsigned intervals and french notations for open intervals to avoid confusion with points in $\Delta^+$.

\section{The category of Mayer-Vietoris systems over $\R$}
\subsection{Middle-exact persistence modules over $\Delta^+$}

We define $\Delta^+ := \{(x,y)\mid x+y > 0\}\subset \R^2$, equipped with the product (partial) order $(x,y)\leq (x',y')$ if and only if $x\leq x'$ and $y\leq y' $. Recall that for $s=(s_1,s_2)\in \R^2_{>0}$, we denote $s_x:=(s_1,0)$ and $s_y := (0,s_2)$. 

\begin{defi}\label{D:PerModule}
A persistence module over $\Delta^+$ is a functor $M : (\Delta^+,\leq) \longrightarrow \Mod(\kk)$, where $\Mod(\kk)$ is the category of $\kk$-vector spaces.

Persistence modules over $\Delta^+$ together with natural transformations of functors form a category, denoted $\Pers(\Delta^+)$.

Similarly the category of persistence comodules over $\Delta^+$ is the category of functors $(\Delta^+,\leq)^{\Op} \longrightarrow \Mod(\kk)$
\end{defi}
In particular, the data of a  persistence module over $\Delta^+$ is encoded by the  structural  morphisms  
\begin{equation}\label{eq:structmorphism}
    M(v) \longrightarrow M(v+s)= M(v_1+s_1, v_2+s_2)
\end{equation}
defined for any $v=(v_1,v_2)\in \Delta^+$ and $s\in \R^2_{>0}$ and their compatibilities. 

\begin{defi} For $s\in \R^2_{>0}$ and $M\in \Obj( \Pers(\Delta^+))$, we define 
$M[s]$ to be the persistence module given, for any $v\in \Delta^+$, by $M[s](v)= M(v+s)$ with structural morphisms induced by those of $M$: for any $t\in \R^2_{>0}$:
$$ M[s](v)=M(v+s) \longrightarrow M(v+s+t)=M[s](v+t).$$
We extend the definition to $s=0$ by $M[0]=M$ with the identity for  $M\to M[0]$.
\end{defi}
It is immediate to check that the structural morphisms~\eqref{eq:structmorphism}
induces, for any $t\in \R^2_{>0}$ the canonical translation maps:
\begin{equation}\label{eq:translmorphism}
   \tau_{s}^M: M \longrightarrow M[s].
\end{equation}
We will adopt the convention that when we do not need explicitly a notation for a translation or structural morphism we simply do not labeled it. 

\smallskip

For $M\in \Pers(\Delta^+)$, any $s\in\R^2_{>0} $ induces the short complex : 
\begin{equation}\label{eq:secforPersModule}
M\{s\} = M \longrightarrow M[s_x]\oplus M[s_y] \longrightarrow M[s] 
\end{equation}
where the first map is $\left( \begin{array}{c} \tau^M_{s_x}\\ -\tau^M_{s_y}\end{array}\right)$ and the second one is $(\tau^{M[s_x]}_{s_y}, \tau^{M[s_y]}_{s_x})$ in matrix notations. In other words for any $t\in \Delta^+$ and  $v\in M(t)$, the first map is given by $v\mapsto (\tau^M_{s_x}(v), - \tau^M_{s_y}(v))$ and the second by 
$M(t+s_x)\oplus M(t+s_y) \ni (v,w) \mapsto \tau^{M[s_x]}_{s_y}(v)+\tau^{M[s_y]}_{s_x}(w)$. 
The fact that~\eqref{eq:secforPersModule} is a complex is an immediate consequence of definition~\ref{D:PerModule}.

\begin{defi}\label{D:middleexact}
$M\in \Obj(\Pers(\Delta^+)$ is said to be \textbf{middle-exact} if the complexes $M\{s\}$ are exact for every $s\in\R^2_{>0} $.
\end{defi}
\begin{remark}
We think of middle-exact complexes as being the analogue for the poset $\Delta^+$ of half the terms of the Mayer-Vietoris long exact sequence relating the various homology groups of two open subsets of a space, their reunion and intersection. What is missing to have a long exact sequence are precisely the connecting homomorphisms relating homology groups of different degrees. In Section~\ref{S:ClassMVSystems}, we will precisely introduce an additional data on a (graded) middle-exact object of $\Pers(\Delta^+)$ to obtain such long exact sequences. 
\end{remark}
Persistence modules have a barcode decomposition similar to peristence modules over $\R$ that we now describe. First we specify the various geometric types, called blocks, of the barcode. 
\begin{defi}\label{def:block_MV}
A block $B$ is a subset of $\R^2$ of the following type : 
\begin{enumerate}
    \item A \textbf{birthblock} (\textbf{bb} for short) if there exists $(a,b)\in \R^2$ such that $B = <a,\infty> \times <b,\infty>$, where $a$ and $b$ can eventually worth $-\infty$ simultaneously. Moreover, we will write that $B$ is of type \textbf{bb$^+$} if $a+b > 0$, and of type \textbf{bb$^-$} if $a+b \leq 0$. 
    \item A \textbf{deathblock} (\textbf{db} for short) if there exists $(a,b)\in \R^2$ such that $B = <-\infty,a> \times <-\infty,b>$.
     \item A \textbf{horizontalblock} (\textbf{hb} for short) if there exists $a\in \R$ and $b\in \R\cup \{+\infty\}$ such that $B = \R \times <a,b>$.
     \item A \textbf{verticalblock} (\textbf{vb} for short) if there exists $a\in \R\cup \{+\infty\}$ and $b\in \R$ such that $B = <a,b> \times \R$.
\end{enumerate}
\end{defi}
\begin{remark}
Blocks are defined over the whole $\R^2$ and not just $\R^2_{>0}$.
\end{remark}
We now define the building blocks (that is the indecomposables) of persistence modules over $\Delta^+$.
\begin{defi}\label{Def:blockmodule}
Let $B$ be a block, define the \textbf{block module} associated to $B$ by, for any $s\leq t \in \Delta^+$, : 
$$\kk^B(s) = \begin{cases} \kk \text{~if~}s\in B \\ 0 \text{~else}
\end{cases} ~~~\kk^B(s\leq t) = \begin{cases} \text{id}_\kk \text{~if~} (s,t)\in B^2 \\ 0 \text{~else} \end{cases} $$
\end{defi}

\begin{thm}[Cochoy-Oudot~\cite{CO17}, Botnan-Crawley-Boevey~\cite{BotCra18}] \label{thm:exactdecomp}
Let $M\in \Pers(\Delta^+)$ be middle exact and pointwise finite dimensional (pfd). Then there exists a unique multiset of blocks $\mathbb{B}(M)$ such that : 

$$M \simeq \bigoplus_{B\in \mathbb{B}(M)} \kk^B $$
\end{thm}




\subsection{Classification of Mayer-Vietoris systems over $\R$}\label{S:ClassMVSystems}

\begin{defi}\label{D:MVSystem}
We define the category $\text{M-V}(\R)$ of \textbf{Mayer-Vietoris persistent systems over} $\R$ as follows : 

\begin{itemize}
    \item[$\bullet$] Objects : collections $S=(S_i,\delta_i^s)_{i\in \Z,s\in \R_{>0}^2}$ where $S_i$ is in $\text{pers}(\Delta^+)$ and $\delta_i^s\in \Hom_{\Delta^+}(S_i[s], S_{i-1})$, such that for all $i\in \Z$ and all $s\in \R_{>0}^2$, the following sequence is exact : 
    \begin{equation}\xymatrix{S_{i+1}[s] \ar[r]^-{\delta_{i+1}^s} & S_i \ar[r] & S_i[s_x] \oplus S_i[s_y] \ar[r] & S_i[s] \ar[r]^-{\delta_{i}^s} & S_{i-1}}\label{eq:MVSystemles}\end{equation}
    
    And the diagram below is commutative, for $s'\geq s$ :  \begin{equation}\label{D:ComSaquareforMV}\xymatrix{  S_i[s] \ar[r]^-{\delta_{i}^{s}} \ar[d] & S_{i-1} \ar[d]^{\text{id}_{S_{i-1}}} \\
    S_i[s'] \ar[r]^-{{\delta}_{i}^{s'}} & S_{i-1}}\end{equation}
    
    \item[$\bullet$] Morphisms : for $(S_i,\delta_i^s)$ and $(T_i,\tilde{\delta}_i^s)$ two Mayer-Vietoris systems over $\R$, a morphism from $(S_i,\delta_i^s)$ to $(T_i,\tilde{\delta}_i^s)$ is a collection of morphisms $(\varphi_i)_{i\in \Z}$ where $\varphi_i \in \Hom_{\Pers(\Delta^+)}(S_i,T_i)$ such that the following diagram
    
   \begin{equation}\xymatrix{  S_i[s] \ar[r]^-{\delta_{i}^s} \ar[d]_{\varphi_i[s]} & S_{i-1} \ar[d]^{\varphi_{i-1}} \\
    T_i[s] \ar[r]^-{\tilde{\delta}_{i}^s} & T_{i-1}}\end{equation} commutes for all $i\in \Z$ and $s\in \R_{>0}^2$. 
\end{itemize}
For a Mayer-Vietoris $S$ and $i\in \Z$, we will write $S_i$ for the associated object of $\Pers(\Delta^+)$ of $S$ which lies in degree $i$.
\end{defi}
A natural class of examples of such M-V systems is provided by homology of level-sets of a  continuous function on a topological space $X$. See, example~\ref{Ex:MVfromFct} below.
\begin{remark}
\begin{itemize}
    \item Observe that if $(S_i,\delta_i^s)$ is a Mayer-Vietoris system, $S_i$ is in particular a middle exact modules, for $i\in \Z$. 
    \item The category $\text{M-V}(\R)$ is indeed a category. It is easy from the definition to observe that it is further additive. However, as we shall see later on, it is not abelian. 
\end{itemize}
\end{remark}

Our remaining goal in this section is to classify Mayer-Vietoris system in a way similar to Theorem~\ref{thm:exactdecomp}. For this, we introduce building blocks for those.
\begin{defi}\label{D:blocksmodulesforMV}
Let $\text{B}$ be a block and $j\in \Z$, then define $\text{S}^\text{B}_j$ the Mayer-Vietoris system associated to $\text{B}$ of degree $j$ by : 

\begin{itemize}
    \item If B is of type \textbf{bb}$^-$, \textbf{hb} or \textbf{vb} then $\text{S}^\text{B}_j = (M_i,0)_{i,s}$ with $M_i = 0$ for all $i\not = j$ and $M_j = \kk_B$
    
    \item If B is of type \textbf{db}, then $\text{S}^\text{B}_j = (M_i,\delta_i^s)$ with $M_i = 0$ for all $i\not \in \{j+1,j\} $, $\delta_i^s=0$ for all $s\in \R^2$ and $i\not = j+1$,  $M_{j+1} = \kk_{B^\dag}$, $M_{j}= \kk_{B}$, and define $\delta_{j+1}^s : \kk_{B^\dag}[s] \to \kk_B$ by point-wise identities on $B^\dag [s]\cap B \cap \Delta^+$.

    \item Dually, if $B$ is of type \textbf{bb$^+$}, then define $S^{\text{B}}_j$ as $S^{B^\dag}_{j-1}$.

    \end{itemize}  
\end{defi}


\begin{rem}
One can easily see that the Mayer-Vietoris systems $S_j^B$ are indecomposables.
\end{rem}

Denote by M-V$^- (\R)$ the full sub-category of Mayer-Vietoris systems over $\R$ whose objects are the MV systems $S$ such that there exists $N\in\Z$ such that $S_i = 0$ for all $i\geq N$. In other words, M-V$^- (\R)$ is the subcategory of bounded below Mayer-Vietoris systems.

\begin{thm}[Classification of pfd M-V systems]\label{thm:decom_MV}

Let $S$ an object of M-V$^- (\R)$ which is pointwise finite dimensional.  Then there exists a unique collection of multisets of blocks $\mathbb{B}(S) = (\mathbb{B}_j(S))_{j\in \Z}$ of type \textbf{bb$^-$}, \textbf{hb}, \textbf{vb}, and \textbf{db}, such that we have an isomorphism in M-V$(\R)$ : 

$$S \simeq \bigoplus_{j\in \Z} \bigoplus_{B\in \mathbb{B}_j(S)} S_j^B $$

$\mathbb{B}(S)$ is the \textbf{barcode} of $S$. It completely determines $S$ up to isomorphism of Mayer-Vietoris systems.
\end{thm}

\begin{remark}
By Definition~\ref{def:block_MV}, birth blocks of type \textbf{bb$^+$} generate the same MV systems as their dual death blocks, therefore  they come in pairs in the decomposition given by Theorem~\ref{thm:decom_MV}, which explains why the blocks of type \textbf{bb$^+$} are ignored in the barcode.
\end{remark}
NEED TO HAD A FIGURE TO SHOW HOW THE MAPS AND EXACT SEQUENCE WORKS OUT FOR SOME BLOCKS.

To prove the theorem~\ref{thm:decom_MV}, we will use the following technical lemmas :

\begin{lem}
Let $S$ be a pfd MV-system over $\R$, such that $\mathbb{B}(S_j)$ contains only blocks of type \textbf{db}, for all $j\in \Z$. Then $S = 0$.

\end{lem}

\begin{pf}
Let $S = (S_j,\delta_j^s)$ be  a pfd MV-system over $\R$, such that $\mathbb{B}(S_j)$ contains only blocks of type \textbf{db}, for all $j\in \Z$. Then, for $s\in \R^2_{s>0}$ and $j\in\Z$, by universal property of cokernels and exactness of~\eqref{eq:secforPersModule}, $\delta^s_j$ factorizes through $$\text{coker}\left (S_j[s_x] \oplus S_j[s_y] \longrightarrow S_j[s] \right ).$$
But this cokernel is trivial 
since $S_j$ is isomorphic to a direct sum of blocks of type \textbf{db}. Therefore, $\delta^s_j = 0$.

Consequently, for every $B\in \mathbb{B}(S_j)$ and every $s\in \R^2_{s>0}$, the exact sequence of persistence modules 
$$0 \longrightarrow \kk_B \longrightarrow \kk_B[s_x] \oplus \kk_B[s_y] $$
yields $B = \emptyset$ since $B$ is assumed to be of type \textbf{db}.
\end{pf}



\begin{lem}
Let $S$ be a pfd MV-system over $\R$, such that there exists  a block $B$ of type \textbf{bb$^+$} such that $B\in\mathbb{B}(S_{j})$, with $j\in\Z$. Then there exists a pfd MV system $\Sigma$ such that : 

$$ S \simeq S_{j}^{B} \oplus \Sigma = S_{j-1}^{B^\dag} \oplus \Sigma  $$
\end{lem}

\begin{pf}
Let $s \geq \sqrt{2}(a+b,a+b)$, since $\text{coker}\left (\kk_B[s_x] \oplus \kk_B[s_y] \longrightarrow \kk_B[s] \right ) \simeq \kk_{B^\dag} $  we have the following commutative diagram, where the rows are exact sequences and where $\varphi$ exists (and is injective) by the universal property of cokernels: 
 
 \begin{equation}\label{eq:MV-summand}
 \begin{gathered}
 \xymatrix{0 \ar[r] & \kk_B \ar@{^{(}->}[d] \ar[r] &\kk_B[s_x] \oplus \kk_B[s_y] \ar[r] \ar@{^{(}->}[d] & \kk_B[s]\ar[r] \ar@{^{(}->}[d] & \kk_{B^\dag} \ar[r] \ar@{.>}[d]^\varphi  & 0 \\
\dots \ar[r] & S_j \ar[r] & S_j[s_x] \oplus S_j[s_y] \ar[r] & S_j[s] \ar[r]^{\delta_j^s} & S_{j-1} \ar[r]& \dots  }
\end{gathered}
\end{equation}
 
Since $B^\dag$ is a directed ideal of $\Delta^+$, $\kk_{B^\dag}$ is an injective object of $\Pe(\Delta^+)$ by lemma 2.1 of\cite{BotCra18}. Therefore, $\varphi$ splits and $\text{im} \varphi \simeq \kk_{B^\dag}$ is a summand of $S_{j+1}$.  The commutativity of~(\ref{eq:MV-summand}) then implies the existence of a complement $X_j$ of $\text{im} (\kk_b\hookrightarrow S_j)$ in $S_j$, such that $S$ decomposes locally as follows:

$$
\xymatrix{X_j \oplus \kk_B \ar^-{\simeq}[d]\ar[r] &X_j[s_x] \oplus X_j[s_y] \oplus  \kk_b[s_x] \oplus \kk_b[s_y] \ar^-{\simeq}[d]\ar[r] & X_j[s] \oplus \kk_B[s] \ar^-{\simeq}[d] \ar[r]  & X_{j-1} \oplus \kk_{B^\dag} \ar^-{\simeq}[d]\\
S_j \ar^-{\sigma}[r] & S_j[s_x] \oplus S_j[s_y] \ar[r] & S_j[s] \ar[r]^{\delta_j^s} & S_{j-1}
}$$

Note that we may assume without loss of generality that $X_j \supseteq \ker \sigma$.
Then, by exactness of~$S$, we have $\text{im} \delta_{j+1}^s = \ker \sigma\subseteq X_j$, therefore our local decomposition extends to a full decomposition of $S$, which means that the upper row complex in~(\ref{eq:MV-summand}) is a summand of~$S$.
\end{pf}

\begin{pf}[of theorem~\ref{thm:decom_MV}] Let $S =(S_i,\delta_i^s)$ be a pointwise finite dimensional Mayer-Vietoris system. In particular, all the $S_i$ are middle-exact pointwise finite dimensional persistence modules over $\Delta^+$. Hence, they decompose uniquely up to isomorphism as a direct sum of block modules by theorem \ref{thm:exactdecomp}.
\begin{itemize}
    \item[The finite barcode case :] We first show the result in the case where $\mathbb{B}(S_j)$ is finite for every $j\in\Z$.  For $j\in\Z$, fix an isomorphism $\varphi_j : S_j \stackrel{\sim}{\longrightarrow}\bigoplus_{B\in\mathbb{B}(S_j)}\kk_B$. Thus, the family $(\varphi_i)_i$ induces an isomorphism of MV systems from $S$ to 
    $$S' := \left(\bigoplus_{B\in\mathbb{B}(S_i)}\kk_B, ~\varphi_{i-1}\circ \delta_i^s \circ \varphi_i^{-1}[s] \right )_{i\in\Z,s\in \R_{>0}^2}$$ 
  
  Let $B\in \mathbb{B}(S_j)$ of type either \textbf{bb}$^-$, \textbf{hb} or \textbf{vb}, then for $s\in \R_{>0}^2$, the map :  $$\kk_B[s_x]\oplus\kk_B[s_y]\longrightarrow \kk_B[s]$$ is surjective. Thus,  $\varphi_{j-1}\circ \delta_j^s \circ \varphi_j^{-1}[s]$ is zero on $\kk_B[s]$. This proves that $S^B_j$ is a summand of $S'$. Finally, with $\mathbb{B}^{-}(S_j)$ the multi-set of intervals of $\mathbb{B}(S_j)$ of type either \textbf{bb}$^-$, \textbf{hb} or \textbf{vb}, we have : 
  
 $$S' = \left (\bigoplus_{j\in\Z} \bigoplus_{B\in  \mathbb{B}^{-}(S_j) } S^B_j \right ) \oplus \left ( \bigoplus_{B\in\mathbb{B}(S_i)\backslash \mathbb{B}^{-}(S_i)}\kk_B, ~\varphi_{i-1}\circ \delta_i^s \circ \varphi_i^{-1}[s] \right )$$ 
 
 Now there remains to prove that the right side of the direct sum, $S''$, decomposes in MV($\R$). For $j\in\Z$, $B\in\mathbb{B}(S_i'')\backslash \mathbb{B}^{-}(S_i'')$ contains only blocks of type either $\textbf{bb$^+$}$ or $\textbf{db}$. We denote by $\mathbb{B}(S_i'')^+$ the multiset of intervals of type \textbf{bb$^+$} of $\mathbb{B}(S_i'')$.
 We will prove by a descending induction on $i\leq 1$, that there exists a MV system $\Sigma^i$ such that : 
 
 $$S'' \simeq \left ( \bigoplus_{i\leq j\leq 1} \bigoplus_{B\in \mathbb{B}(S_j'')^+ } S_j^B \right ) \oplus \Sigma^i $$
 
 For $i = 1$ the property clearly holds with $\Sigma^i = S''$.
 Let us now assume the decomposition for $i\leq 1$. We now prove by induction on the cardinality of $\mathbb{B}(S_{i-1}'')^+$, that each block $B$ appearing in this multiset, leads to a summand of type $S_{i-1}^B$. Let $B$ such a block, with $B = <a,\infty> \times <b,\infty>$ where $(a,b) \in \Delta^+$. For $s \geq \sqrt{2}(a+b,a+b)$  we have the following commutative diagram : 
 
 $$\xymatrix{0 \ar[r] & \kk_B \ar@{^{(}->}[d] \ar[r] &\kk_B[s_x] \oplus \kk_B[s_y] \ar[r] \ar@{^{(}->}[d] & \kk_B[s]\ar[r] \ar@{^{(}->}[d] & \kk_{B^\dag} \ar[r] \ar@{.>}[d]^\varphi  & 0 \\
\dots \ar[r] & \Sigma^i_{i-1} \ar[r] & \Sigma^i_{i-1}[s_x] \oplus \Sigma^i_{i-1}[s_y] \ar[r] & \Sigma^i_{i-1}[s] \ar[r] & \Sigma^i_i \ar[r]& \dots  } $$
 
 Where the dotted arrow $\varphi$ exists by the universal property of cokernels, and it is clear that $\varphi$ is a monomorphism. Since $B^\dag$ is a directed ideal of $\Delta^+$, $\kk_{B^\dag}$ is an injective object of $\Pe(\Delta^+)$ by lemma 2.1 of\cite{BotCra18}. Therefore, $\varphi$ splits and $\text{im} \varphi \simeq \kk_{B^\dag}$ is a summand of $\Sigma^i_i$. The commutativity of the diagram then shows that the upper row chain complex is a summand of the lower row. The supplementary summand of this complex have a barcode in degree $i-1$ with strictly less blocks of type \textbf{bb$^+$}, which conclude the induction step.
 
 Now for $j\geq i$, the barcode of $\Sigma_{j}^i$ can contain only deathblocks by construction. Then, writting the long exact sequence given by the connection morphisms of $\Sigma^i$ for a suitably choosen $s\in\R_{>0}^2$ proves that $\Sigma_j^i=0$. \textcolor{red}{rajouter preuve détaillée en lemme}.
 
 Therefore, we can finally conclude that :
 
 $$S' \simeq \bigoplus_{j\in\Z} \bigoplus_{B\in \mathbb{B}(S'_j)^+} S^B_j$$
 
 \item[Generalization to the infinite barcode case :]
We now generalize to the case where $S_i$ can have an infinite barcode. For the same reason as in the finite case, each block of type \textbf{bb}$^-$, \textbf{hb} or \textbf{vb} of $S_i$ splits as a summand $S_i^B$ of $S$. Hence, we are reduced to prove the existence of the decomposition in the case where $S$ is a pfd MV system, and $\mathbb{B}(S_j)$ contains only blocks of type \textbf{bb$^+$} or \textbf{db}, for all $j\in\Z$. For $n\in \Z_{>0}$, define $\Delta^+_n := \Delta^+\cap\{(x,y)\in \R^2 \mid  x\leq n , y\leq n\}$. Define : 
$$\mathbb{B}(S_j)_n := \{B\in \mathbb{B}(S_j) \mid B ~\text{is of type \textbf{bb$^+$} and } B\cap \Delta^+_n \not = \emptyset ~ \text{or~} B \text{~is of type \textbf{db} and } B \subset \Delta^+_n \}  $$

Then it is clear that $\mathbb{B}(S_j) = \cup_n \mathbb{B}(S_j)_n$, and since $S$ is pointwise finite dimensional, $\mathbb{B}(S_j)_n$ contains finitely many blocks of type \textbf{bb$^+$}, for all $n \geq 0$.
We now identify each $S_j$ with its block decomposition by some fixed isomorphisms, and for $n\geq 0$ we define ${}_n\tilde{S}$ to be the collection : 

$${}_n\tilde{S} = \left (\bigoplus_{B\in \mathbb{B}(S_j)_n } \kk_{B}, (\delta_j^s)_{|\bigoplus_{B\in \mathbb{B}(S_j)_n } \kk_{B}} \right )  $$

 Let us prove that $\tilde{S}$ is a sub-MV system of $S$. To do so, it is sufficient to prove that for all $j\in  \Z$, the image of $(\delta_j^s)_{|\bigoplus_{B\in \mathbb{B}(S_j)_n}}$ is contained in $\bigoplus_{B\in \mathbb{B}(S_{j-1})_n } \kk_{B}$. Fix $j\in \Z$ and $s\in \R^2_{>0}$. Then $(\delta_j^s)_{|\bigoplus_{B\in \mathbb{B}(S_j)_n}}$ factorizes uniquely through : 
 
 \begin{align*}
     & \text{coker} \left ( \bigoplus_{B\in \mathbb{B}(S_j)_n } \kk_{B}[s_x] \oplus  \kk_{B} [s_y] \longrightarrow \bigoplus_{B\in \mathbb{B}(S_j)_n } \kk_{B}[s]  \right )   \\ 
     &\simeq \bigoplus_{B\in \mathbb{B}(S_j)_n }  \text{coker} \left ( \kk_{B}[s_x] \oplus  \kk_{B} [s_y] \longrightarrow  \kk_{B}[s] \right ) \\
     &=  \bigoplus_{ \substack{B\in \mathbb{B}(S_j)_n\\ B ~\text{is of type \textbf{bb$^+$}} } } \text{coker} \left ( \kk_{B}[s_x] \oplus  \kk_{B} [s_y] \longrightarrow  \kk_{B}[s] \right ) 
 \end{align*} 
 
As previously, for every $B\in \mathbb{B}(S_j)_n$ of type \textbf{bb$^+$}, we can find $s\in \R^2_{>0}$ such that the canonical map : 

$$ \text{coker} \big ( \kk_{B}[s_x] \oplus  \kk_{B} [s_y] \longrightarrow  \kk_{B}[s] \big  )  \longrightarrow \bigoplus_{B\in \mathbb{B}(S_{j-1}) } \kk_{B} $$

is a monomorphism. By the same argument as in the previous case, as  $ \text{coker} \big ( \kk_{B}[s_x] \oplus  \kk_{B} [s_y] \longrightarrow  \kk_{B}[s] \big  ) $ is an injective object of $\Pers(\Delta^+)$, its image splits as a summand of $\bigoplus_{B\in \mathbb{B}(S_{j-1}) } \kk_{B}$, hence, is a subset of 
$$  \kk_{B^\dag}^m\subset \bigoplus_{B\in \mathbb{B}(S_{j-1}) } \kk_B $$

where $m$ is the multiplicity of $B^\dag$ in $\mathbb{B}(S_{j-1}) $. As $B^\dag\in \mathbb{B}(S_{j-1})_n$, we conclude that $\text{im} ((\delta_j^s)_{|\bigoplus_{B\in \mathbb{B}(S_j)_n}}) \subset \bigoplus_{B\in \mathbb{B}(S_{j-1})_n } \kk_{B} $. This proves that ${}_n\tilde{S}$ is a sub-MV system of $S$.

Moreover, we can apply our first case result to ${}_n\tilde{S}$. Since : 

$$S = \bigcup_{n\geq 0} {}_n\tilde{S} $$

and that this filtration stabilizes point-wise, we deduce the decomposition for $S$.


\end{itemize}

\end{pf}

%Definition of the category. Classification. Barcodes. Representation of the band (cf. Magnus)

\subsection{Interleaving distance for M-V systems}
We have a (fully faithfull) functor $(\R ,\leq) \to (\R^2, \leq)$ given by the diagonal embedding  $t\mapsto \vec{t}$ where $\vec{t}=(t,t)$. We also denote  the functor $\vec{(-)}: (\R_{>0} ,\leq) \to (\Delta^+, \leq)$  the induced functor. 

\smallskip

Given $\varepsilon \geq 0$, and $M = (M_i,\delta_i^s)$ a Mayer-Vietoris system over $\R$ (Definition~\ref{D:MVSystem}), observe that the collection $\tau_{\vec{\varepsilon}}^M:=(\tau_{\vec{\varepsilon}}^{M_i})_{i\in\Z}$ is a morphism of Mayer-Vietoris sytems $M \longrightarrow M[\vec{\varepsilon}]$, where $M[\vec{\varepsilon}]:=(M_i[\vec{\varepsilon}],\delta_i^s[\vec{\varepsilon}])$.

\begin{defi}
Let $M$ and $N$ two Mayer-Vietoris systems over $\R$. An \textbf{$\varepsilon$-interleaving} between $M$ and $N$ is the data of two morphisms of M-V systems $f = (f_i) : (M_i,\delta_i^s) \longrightarrow (N_i[\vec{\varepsilon}],\tilde{\delta}_i^s [\vec{\varepsilon}])$ and $g = (g_i) : (N_i,\tilde{\delta}_i^s) \longrightarrow (M_i[\vec{\varepsilon}],\delta_i^s[\vec{\varepsilon}]) $ such that the following diagram commutes : 

\begin{equation}\label{eq:Interleaving} \xymatrix{
M  \ar[rd]\ar@/^0.7cm/[rr]^{\tau_{2\vec{\varepsilon}}^M} \ar[r]^{f} & N[\vec{\varepsilon}]  \ar[rd] \ar[r]^{g[\vec{\varepsilon}]} & M[2\vec{\varepsilon}] \\
N  \ar[ur]\ar@/_0.7cm/[rr]_{\tau_{2\vec{\varepsilon}}^N} \ar[r]^{g} & M[\vec{\varepsilon}]  \ar[ur] \ar[r]^{f[\vec{\varepsilon}]} & N[2\vec{\varepsilon}]
   } \end{equation}
   
If $M$ and $N$ are $\varepsilon$-interleaved, we shall write $M\sim_\varepsilon^{MV} N$

\end{defi}



\begin{defi}\label{D:InterleavingforMV}
Define the interleaving distance between two Mayer-Vietoris systems $M$ and $N$ to be the positive or possibly infinite number : 

$$d_I^{MV}(M,N) := \inf \{\varepsilon \geq 0 \mid M \sim_\varepsilon^{MV} N  \}. $$
\end{defi}
\begin{remark}\label{R:DMVgeneralizedDstandard}
 The interleaving distance for Mayer-Vietoris system is just the graded extension of the usual interleaving distance in $\Pe(\Delta^+)$ defined, for $M, N \in \Obj(\Pe(\Delta^+))$ by 
 $$d_I(M,N) := \inf \{\varepsilon \geq 0 \mid M \sim_\varepsilon^{\Delta^+} N  \} $$
 where $M \sim_\varepsilon^{\Delta^+} N$ means that $M$ and $N$ are $\varepsilon$-interleaved as persistence modules, that is there  exists $f: M\to N[\vec{\varepsilon}]$ and $g: N\to M[\vec{\varepsilon}]$ are persistence modules morphisms satisfying that the diagram~\eqref{eq:Interleaving} commutes. 
 
\end{remark}

 We say that a Mayer-Vietoris system $M=(M_i,\delta_i)_{i\in \Z}$ is \textbf{bounded} if there is only finitely many $M_i$ which are non-zero.
 
 
 

Let us denote $B -\vec{\varepsilon}=\{ s -\vec{\varepsilon}, \, s\in B\}$; this is a block of the same type as $B$.
\begin{lem}\label{L:shiftofBlock}
 Let $B$ a block and $\varepsilon >0$. There is a canonical isomorphism 
 $$\kk^B[\vec{\varepsilon}] \, \cong \, \kk^{B-\vec{\varepsilon}}. $$
 If $B'$ is another block such that $B-\vec{\varepsilon} \subset B'$ and $B'-\vec{\varepsilon} \subset B$, then $$\kk^B \sim^{\Delta^+}_{\varepsilon} \kk^{B'}.$$
\end{lem}
\begin{pf}
 By definition~\ref{Def:blockmodule}, we have that 
 $$\kk^B[\vec{\varepsilon}](t)= \kk^B(t+(\varepsilon, \varepsilon) = \left\{ \begin{array}{ll} 
\kk & \text{if t }\in B -\vec{\varepsilon} \\
0 & \text{else.}\end{array}\right.$$
Therefore we have that $$\kk^B[\vec{\varepsilon}] \, \cong \, \kk^{B-\vec{\varepsilon}}.$$

TO COMPLETE (AND CHECK IT IS THE CORRECT INCLUSIONS)
\end{pf}

We now turn to a main source of examples of Mayer-Vietoris systems. 
\begin{ex}[Mayer-Vietoris system associated to continuous functions]\label{Ex:MVfromFct}
Let $h: X\to \R$ be a continuous function on a topological space $X$. For any $
x=(x_1,x_2)\in \Delta^+$, we 
 set $$M_i^h(x):= H_i(h^{-1}(]-x_1, x_2[).$$
 If $x'=(x'_1, x'_2)\geq x$, then we have the inclusion $]-x_1, x_2[\subset ]-x_1', x_2'[$ inducing, for all $i$'s, homomorphisms $M_i^h(x)=H_i(h^{-1}(]-x_1, x_2[) \to H_i(h^{-1}(]-x'_1, x'_2[)= M^h(x')$ in homology. 
 By Lemma~\ref{L:Defiota}, this makes $M_i^h(-)$ a persistence module over $\Delta^+$, which is call the \textbf{level set persistence module associated to $f:X\to \R$}. 
 
 \smallskip
 
 Now, let $s(s_1,s_2) \in \R^2_{>0}$. For any $x=(x_1,x_2)\in \Delta^+$, we have that the open interval $]-x_1-s_1, x_2,+s_2[ $ has a cover given by the two open sub-intervals $]-x_1-s_1, x_2[$ and $]-x_1, x_2,+s_2[ $ whose intersection is $]-x_1, x_2[$. Therefore the Mayer-Vietoris sequence associated to this cover gives us linear maps $(\delta_{i}^{s,x})_{i\in \N}$ and exact sequences 
  \begin{equation}\xymatrix{M^h_{i+1}[s](x) \ar[r]^-{\delta_{i+1}^{s,x}} & M^h_i(x) \ar[r] & M^h_i[s_x](x) \oplus M^h_i[s_y](x) \ar[r] & M^h_i[s](x) \ar[r]^-{\delta_{i}^{s,x}} & M^h_{i-1}}(x).\label{eq:MVSystemfromFct}\end{equation}
  We write $\delta^s_i: M_i^h[s] \to M_{i-1}^h$ the maps given at every point $x$ by $\delta_i^{s,x}$ and for $i\leq 0$ we set $\delta^i=0$.
  \begin{prop}\label{P:MVfromFct}
  The  $\delta^s_i$'s are persistence modules maps and further makes $(M^h_i, \delta^s_i)_{i\in \Z, s\in \R^2_{>0}}$ a Mayer-Vietoris system over $\R$.
  \end{prop}
  \begin{pf}
  The fact that the $\delta^s_i$ are persistence modules maps as well as the commutativity of diagram~\eqref{D:ComSaquareforMV} follow from the naturality of the Mayer-Vietoris sequence. The exactness of~\eqref{eq:MVSystemfromFct} implies the condition~\eqref{eq:MVSystemles}.
  \end{pf}
  We call  $(M^h_i, \delta^s_i)_{i\in \Z, s\in \R^2_{>0}}$, the \textbf{level set Mayer-Vietoris system associated to $f:X\to \R$}. 
\end{ex}


In particular, we obtain from the degree-wise stability of interleaving distance between level-set persistence modules the following : 

\begin{prop}\label{P:StabilityfordMV}
Let $h_1,h_2 : X \to \R$ two continous functions defined on the topological space $X$. Then : $$d_I^{MV}(M^{h_1},M^{h_2})\leq \sup_{x\in X} |h_1(x)-h_2(x)| $$

\end{prop}
\begin{pf} If the distance is $\infty$, there is nothing to prove. Otherwise, let $\varepsilon= \sup_{x\in X} |h_1(x)-h_2(x)|$. Then for any $(x,y)\in \Delta^{+}$, we have  level-set inclusions  $h^{-1}_1(]-x,y[) \subset h^{-1}_2(]-x-\varepsilon, y+\varepsilon[)$ and  $h^{-1}_2(]-x,y[) \subset h^{-1}_1(]-x-\varepsilon, y+\varepsilon[)$ which induce persistence modules over $\Delta^+$ morphisms 
$$f:\big(M_i^{h_1}(x,y)=H_i(h_1^{-1}(]-x,y[) \to H_i(h_2^{-1}(]-x-\varepsilon,y+\varepsilon[)= M_i^{h_2}[\vec{\varepsilon}](x,y)\big)_{(x,y)\in \Delta^+},  $$ 
$$g:\big(M_i^{h_2}(x,y)=H_i(h_2^{-1}(]-x,y[) \to H_i(h_1^{-1}(]-x-\varepsilon,y+\varepsilon[)= M_i^{h_1}[\vec{\varepsilon}](x,y)\big)_{(x,y)\in \Delta^+} $$ since 
taking homology groups is a functor and by lemma~\ref{L:Defiota}. 

The fact that these maps are Mayer-Vietoris systems morphisms follows again as in proposition~\ref{P:MVfromFct} by the naturality of the Mayer-Vietoris sequence associated to open covers of the intervals $]-x-\varepsilon, y+\varepsilon[$ by $]-x-\varepsilon, y[$ and $]-x, y+\varepsilon[$.
\end{pf}


\section{Stable sheaf theoretic interpretation of level-set persistence}
In this section we follow the standard notations of \cite{Kash90}, \cite{KS18}, \cite{Berk18} for sheaves. 
In particular,  $\kk$ will denote a field, $\Mod(\kk)$ the category of vector spaces over $\kk$ and, for   a topological space $X$,  we will note $\Mod(\kk_X)$ the category of sheaves of $\kk$-vector spaces on $X$ and $\text{PSh(X)}$ the category fo presheaves of $\kk$-modules on $X$. For shortness, we will also write $\Hom$ for $\Hom_{\Mod(\kk_\R)}$.

Further, $\D^b(\kk)$ will be the \emph{derived} category of complexes of $\kk$-modules with bounded cohomology and $\D^b(\kk_X)$ will be the one of complexes of sheaves with bounded cohomology of $\kk$-modules over $X$. 
 Unless the context is unclear, we will simply use the word  sheaf for an object of $\D^b(\kk_X)$. 
 
We will use the standard Grothendieck operations on sheaves as in~\cite{Kash90}.  
 
 CITE CURRY ALSO AND BACKGROUND ADD IDEAS on WHAT IS $\kk_S $ FOR OPEN-CLOSED SETS
 
\subsection{Convolution distance for sheaves after Kashiwara-Schapira}
In~\cite{KS18} Kashiwara and Schapira have defined a (pseudo)distance on the derived category of sheaves. This distance is a sheaf, and derived by design, version of the interleaving distance of persistence modules. It is based on convolution of sheaves which we now explain. 

Let  $\V$  be an euclidean vector space, which in our case of interest will simply be $\Rr$. We let $s: \V\times \V \to \V$ be the addtion map $(t,t')\mapsto t+t'$.  

\begin{defi}\label{D:Convolution} The convolution of sheaves  $\D^b(\kk_\V)\times \D^b(\kk_\V) \to \D^b(\kk_\V)$ is the bifunctor given, 
for $F,G\in \Obj(\D^b(\kk_\V))$,  by the formula : $$F\star G = R s_!(F\boxtimes G)$$ where $\boxtimes$ is the external tensor product of sheaves. 
\end{defi}
To define the convolution distance,  we will only need a very specific case : the convolution by the constant sheaf supported on a ball centered at 0. 
More precisely we define, for $\varepsilon \in \Rr$, 
\begin{equation}\label{eq:defKepsilon}
 K_\varepsilon :=  \left\{ \begin{array}{cc} 
 \kk_{x\in \V, \mid \|x\|\leq \varepsilon}  & \mbox{si} \varepsilon \geq 0 \\
 \kk_{x\in \V, \mid \|x\|< -\varepsilon}[\dim(\V)]  & \mbox{si} \varepsilon < 0 .\end{array} \right.
\end{equation}
The convolution by $K_\varepsilon$ has some nice properties : 
\begin{prop}\label{P:propertiesofconvolution} Let $\varepsilon, \varepsilon'\in \R$ and $F \in \Obj(\D^b(\kk_\V))$. Then
\begin{enumerate}
\item there are functorial isomorphisms $F\star K_0\simeq F$ and $(F\star K_{\varepsilon} )\star K_{\varepsilon'} \simeq F \star K_{\varepsilon + \varepsilon'} $.
\item If $\varepsilon' \geq \varepsilon $, there is a canonical morphism of sheaves 
$K_{\varepsilon'}\to K_{\varepsilon}$ in $\D^b(\kk_\V)$
inducing a natural transformation $F\star K_{\varepsilon'} \to F \star K_{\varepsilon} $. 
In the special case where $\varepsilon = 0$, we simply write $\phi_{F, \varepsilon'}$ for this natural transformation.
\end{enumerate}
\end{prop}
The convolution by $K_{\varepsilon}$ is the sheaf analogue of the canonical shift $F[\varepsilon]$ of a persistence module. Indeed, by
Proposition~\ref{P:propertiesofconvolution}.(1),  for any map $f: F\star K_{\varepsilon} \to G$ we get  canonical maps  
\begin{equation} \label{eq:propertiesofconvolution} f\star K_{\tau}: F\star K_{\varepsilon+\tau}\simeq F\star  K_{\varepsilon}\star K_{\tau}  \to G \star K_{\tau}.\end{equation} 
The above maps allow us to define interleaving.
\begin{defi}\label{D:inteleavingconvolution}
\begin{enumerate}\item For $F,G\in \Obj(\D^b(\kk_\V))$ and $\varepsilon\geq 0$, one says that $F$ and $G$
are $\varepsilon$-\textbf{interleaved} if there exists two morphisms in $\D^b(\kk_\V)$, $f : F\star K_\varepsilon \to G$ 
and $g :  G \star K_\varepsilon \to F$ such that the compositions
$F\star K_{2\varepsilon} \stackrel{ f\star K_\varepsilon}{\longrightarrow}K_\varepsilon\star G \stackrel{g}{\longrightarrow} F $ and $G\star K_{2\varepsilon} \stackrel{g \star K_\varepsilon}{\longrightarrow}K_\varepsilon\star F \stackrel{f}{\longrightarrow} G $ are the natural morphisms $F\star K_{2\varepsilon} \stackrel{\phi_{F,2\varepsilon}}{\longrightarrow} F$ and $G\star K_{2\varepsilon} \stackrel{\phi_{G,2\varepsilon}}{\longrightarrow} G$, that is, we have a commutative diagram in $\D^b(\kk_\V)$ :

$$ \xymatrix{
F\star K_{2\varepsilon}  \ar[rd]\ar@/^0.7cm/[rr]^{\phi_{F,2\varepsilon}} \ar[r]^{f\star K_\varepsilon} & G\star K_\varepsilon  \ar[rd] \ar[r]^{g} & F\\
G \star K_{2\varepsilon}  \ar[ur]\ar@/_0.7cm/[rr]_{\phi_{G,2\varepsilon}} \ar[r]^{g\star K_\varepsilon} & F\star K_\varepsilon  \ar[ur] \ar[r]^{f} & G
   } $$ 
In this case, we write $F \sim_\varepsilon G$.
\item or $F,G\in \Obj(\D^b(\kk_\V))$, we define their \textbf{convolution distance} as : $$d_C(F,G) : = \text{inf} \left (\{ +\infty \} \cup \{a \in \R_{\geq 0 } \mid \text{$F$ and $G$ are $a$-isomorphic} \} \right )$$ \end{enumerate}
\end{defi}
The convolution distance has the following properties
\begin{prop}[\cite{KS18} and \cite{Berk18}]\begin{enumerate}
\item The convolution distance is a closed extended pseudo-metric on $\D^b(\kk_\V)$ that is, for $F,G,H \in \Obj(\D^b(\kk_\V))$ : 
\begin{enumerate}
\item $d_C(F,G) = d_C(G,F)$,
\item $d_C(F,G) \leq d_C(F,H) + d_C(H,G)$,
\item  one has $d_C(F,G) \leq \varepsilon 
\iff F \sim_\varepsilon G $.
\end{enumerate}
\item \textbf{(Stability Theorem)} If $X$ is a locally compact topological space, and $f,g : X \to \V$ are continuous functions, then for any $F\in \Obj(\D^b(\kk_\V))$ one has : $$d_C(\R f_* F, \R g_* F) \leq \sup_{x\in X} \|f(x) - g(x) \|$$
and the same is true for $\R f_!$ and $\R g_!$.
\end{enumerate}
\end{prop}


\subsection{Graded barcodes and derived isometry theorem after~\cite{Berk18}}
There is a notion of barcodes for \emph{constructible} sheaves that mimicks the persistence case. This allow  to define a derived bottleneck distance.


\begin{defi}
A sheaf  $F\in \Obj(\Mod(\kk_\V))$, $F$ is said to be \textbf{constructible} if there exists a locally finite stratification of $\V = \sqcup_\alpha S_\alpha$, such that for each stratum $S_\alpha$ is locally closed in $\V$, the restriction $F_{|S_\alpha}$ is locally constant and further,  the stalks $F_x$ are of finite dimension for every $x\in \V$. 

We  write respecively $\Mod_{\R c}(\kk_M)$ and $\D^b_{\R c}(\kk_M)\cong \D^b(\Mod_{\R c}(\kk_M))$ for the  category of constructible sheaves on $M$ and  the full (triangulated) subcategory of $\D^b(\kk_M)$ consisting of complexes of sheaves whose cohomology objects lies in $\Mod_{\R c}(\kk_M)$.
\end{defi}
Note that the notion of constructibility is precisely what is usually called $\R$-constructibility. Since no other notion will show up in this work we simply drop the $\R$. 
\begin{remark} The condition on the stalks is the sheave counterpart of the condition of being pointwise finite dimensional  for persistence modules.   
\end{remark}


There is a decomposition similar to persistence for constructible sheaves. Namely we have the following two results.
\begin{thm}[Decomposition - \cite{KS18} Theorem 1.17.]\label{T:KSdecomposition}
Let $F \in \Obj(\Mod_{\R c}(\kk_\R))$, then there exists a locally finite family of intervals $\{I_\alpha\}_{\alpha \in A}$ such that $F \simeq \bigoplus_{\alpha \in A} \kk_{I_\alpha}$. Moreover, this decomposition is unique up to isomorphism.
\end{thm}
\begin{cor}[Structure]\label{T:KSstructure}
Let $G^\bullet\in \Obj(\D^b_{\R c}(\kk_\R))$. 
\begin{enumerate} \item Then there exists an isomorphism in $\D^b_{\R c}(\kk_\R)$ : $$G^\bullet \simeq \bigoplus_{j\in\Z} \Ho^j(G^\bullet)[-j]$$
where $\Ho^j(G^\bullet)$ is seen as a complex concentrated in degree $0$.
\item For each $j\in \Z$, there is a unique multiset $\B^j(G^\bullet) $ of intervals such that  $\Ho^j(G^\bullet)\simeq \bigoplus_{I \in \B^j(G^\bullet)} \kk_{I}$. 
\end{enumerate}\end{cor}
The Corollary allows us to define the graded barcode of an object of $\D^b_{\R c}(\kk_\R)$ following~\cite{Berk18}.
\begin{defi}\label{D:gradedbarcode}
 The \textbf{graded-barcode} $\B^\bullet(G^\bullet)$ of $G^\bullet$ is the sequence of multisets $(\B^j(G^\bullet))_{j\in \Z}$.
 
We write $\B_C^\bullet(G^\bullet)$, $\B_L^\bullet(G^\bullet)$ and $\B_R^\bullet(G^\bullet)$ for the  sub-multisets of $\B^\bullet(G^\bullet)$ consisting respectively of the closed or bounded open intervals, semi-open intervals which are open on the right,  semi-open intervals which are open on the right (and not equal to $\R$). 

The interval appearing in the respective substes will be calle respectiveley of central type, left type and right type.
 \end{defi}
By the corollary~\ref{T:KSstructure}, the graded-bracode uniquely determines the complex of sheaves up to isomorphisms in $\D^b_{\R c}(\kk_\R)$ and further we have a unique decomposition $$G^\bullet \cong G_C^\bullet \oplus G_L^\bullet \oplus G_R^\bullet $$ into sheaves whose cohomology only have supports    
in intervals of central type, left type and right type respectively. This is the CLR decomposition of~\cite{Berk18}.
An important point of~\cite{Berk18} is the following:
\begin{lem}[\cite{Berk18}]\label{L:Caracinterleaving} Let $F^\bullet, \, G^\bullet \in \D^b_{\R c}(\kk_\R)$.
\begin{enumerate}
\item If $\varepsilon \in \R$, then $G^\bullet$ and $F^\bullet$ are $\varepsilon$-interleaved if and only if 
 $G_C^\bullet \mathop{\sim}\limits_{\varepsilon} F_C^\bullet$, $G_L^\bullet \mathop{\sim}\limits_{\varepsilon} F_L^\bullet$ and $G_R^\bullet \mathop{\sim}\limits_{\varepsilon} F_R^\bullet$. In particular, 
 the bars of a given type can only be interleaved with a bar of the same type.
 \item Further, an open bar can be $\varepsilon$-interleaved only with a closed bar of degree one more and all other possible $\varepsilon$-interleaving between bars of the same exact types are between bars of same degree. \end{enumerate}
\end{lem}
One can characterize the geometric  condition under which two sheaves $\kk_I$ and $\kk_J$ on intervals of the same type are $\varepsilon$-interleaved, precisely in terms of proximity of the endpoints.

 In order to define the bottleneck distance, we first define the notion of $\varepsilon$-matching. 
 
\begin{defi}\label{D:Epsilonmatching}
Let $\B$ and $\B'$ be two graded-barcodes and $\varepsilon \geq 0$. An $\varepsilon$-\textbf{matching} between $\B$ and $\B$ is the data of
\begin{enumerate}
\item partial matchings: $\sigma_R^j: \B^j_R \not \to \B_R^j$, $\sigma_L^j: \B^j_L \not \to (\B')_L^j$ for all $j\in \Z$
satisfying that 
\begin{description}
\item $(i)$ for any matched pair $I$, $\sigma_R(I)$  (resp. $J$, $\sigma_L (J)$), 
one has $\kk_I[-i] \sim_\varepsilon\kk_{\sigma_R^j(I)}[-i]$ (resp. $\kk_J[-j] \sim_\varepsilon\kk_{\sigma_L^j(J)}[-j]$) and
\item $(ii)$ for  the $I \in \B_R$ and $\B_L$ which are not matched,  one has $\kk_I[-i]\sim_\varepsilon 0$.\end{description}
\item a \emph{bijection} $\sigma^j_C :  \B_C \longrightarrow \B'_C$ satisfying,  for any $I \in B^j_C $, that  $\kk_I \sim_\varepsilon \kk_{\sigma^j_C(I)}[-\delta]$ and further 
\begin{description}
\item $(i)$ that $\delta =0 $ if $I$ and $\sigma^j_C(I)$ are both open or both closed 
\item $(ii)$ and $\delta = 1$ if $I$ is open and $\sigma^j_C(I)$ is closed, $\delta = -1$ if  $I$ is closed and $\sigma^j_C(I)$ is open.
 \end{description}
\end{enumerate}
\end{defi}
\begin{remark}
 An $\varepsilon$-matching can match bars of different degrees, but only if one them is closed and the other is open which differs in degree by $1$. In all other situations one can only match bars of teh same degrees. The reason for this is given by the point 2 of lemma~\ref{L:Caracinterleaving}.
\end{remark}

\begin{defi}
Let $\B$ and $\B'$ be two graded-barcodes, then one defines their \textbf{bottleneck distance} to be the possibly infinite positive value: $$d_B(\B,\B') = \text{inf} \{\varepsilon \geq 0 \mid \text{there exists a } \varepsilon  \text{-matching between } \B \text{ and } \B' \} $$
\end{defi}
The graded bottleneck distance is isometric to the convolution.
\begin{thm}[Isometry~\cite{Berk18}]\label{T:DerivedIsometry}
Let $Fb^\bullet,\, G^\bullet$ be two objects of $\D^b_{\R c}(\kk_\R)$. Then we have 
$$ d_C(F,G) =  d_B(\B(F),\B(G))$$
\end{thm} 
\subsection{Extending level-set persistence modules as pre-sheaves over $\R$}\label{SS:levelsettopresheaves}
In this section we interpret level-set persistence modules as (pre)sheaves on the line $\R$. 


 Let $(\Ouv(\R), \subset )$ be the poset of open subsets of $\R$ ordered by the inclusion. We dentoe in the same way the associated category.  
\begin{lem} \label{L:Defiota}Set $\iota: (\Delta^+, \leq)\to (\Ouv(\R), \subset)$ to be given on objects by 
  $\iota: s=(s_1, s_2) \mapsto (-s_1, s_2)$ where the right hand side is an open interval. 
  Then $\iota$ is well defined and a fully faithfull functor.
  
  The image of $\iota$ is precisely the full subcategory of bounded open intervals of $\R$ 
\end{lem}
In particular, restricting to objects of those categories,  $\iota$ is a bijection from $\R^2$ to bounded open intervals of $\R$. 
\begin{pf}
 By definition  $$ s=(s_1, s_2)\in \Delta^+ \;\Longleftrightarrow \;-s_1<s_2$$ hence $\iota$ is well defined, injective on objects with image the bounded open intervals.  Further if $(s_1, s_2)\leq (s_1', s_2')$ then 
 $-s_1'\leq -s_1<s_2\leq s_2'$ which proves that $\iota$ is order presearving (and necessarily fully faithfull since the morphisms are empty or a singleton). 
\end{pf}



Given $M \in \Obj(\Pe(\Delta^+))$ we can consider its pointwise dual $ t\mapsto \Hom_{\Mod(\kk)}(M(t);\kk)$ which has a canonical structure of a persistence comodule, that is of an object of  
$\text{Fun}((\Delta^{+})^{\op}; \Mod(\kk))\cong \text{Fun}(\Delta^+; \Mod(\kk)^{\op})^{\op} $.  We denote by $M^*$ this dual of $M$. More precisely, $M^*$ is the composition of functors
$$M^* :=\, \Delta^{+\op} \stackrel{M^{\op}}\longrightarrow \Mod(\kk)^{\op} \stackrel{\Hom_{\Mod(\kk)}(-;\kk)}\longrightarrow \Mod(\kk). $$

Since $M^*$ is a persistence comodule, for any  open $U \subset \R$, we can consider the module 
\begin{equation}\label{eq:defMtilde} 
\Tilde{M} (U):=  \varprojlim_{]-x,y[\subset U} M^*((x,y)).\end{equation}
                                     
\begin{lem} There is a functor $\tilde{-}: \Pe(\Delta^+)\to \text{PSh}(\R)^{\op}$   extending the formula~\eqref{eq:defMtilde} into a canonical presheaf on $\R$, that is such that for   $U \in \Obj(\Ouv(\R))$, one has $$\Tilde{M} (U):=  \varprojlim_{]-x,y[\subset U} M^*((x,y)).$$ 
\end{lem}
\begin{pf} One notice that the formula exhibits  $\Tilde{M}$ as a  left Kan extension which makes it into a presheaf canonically. Indeed, 
 consider $\iota^\text{op} : (\Delta^{+})^{\op}\longrightarrow \Ouv(\R)^{\op}$ the (opposite of the) functor defined previously (see~\ref{L:Defiota}) and let  $\text{Lan}_{\iota^\text{op}} M^*$  be the left Kan extension along $\iota^\text{op}$ of $M^*$, which is therefore by definition an object of $\text{PSh}(X)$ : 
$$\xymatrix{ \Delta^{+\text{op}}
\ar@{^{(}->}[r]^-{\iota} \ar[d]_{M^*} & \Ouv(\R)^\text{op} \ar@{.>}[dl]^{\text{Lan}_\iota M^* =: \Tilde{M}} \\
\Mod(\kk) & } $$
As $\Mod(\kk)$ is complete,  the pointwise formula~\eqref{eq:defMtilde} is an immediate consequence.  
%for $U \in \Obj(\Ouv(\R))$ : $$\Tilde{M} (U):= \text{Lan}_\iota M^*(U) = \varprojlim_{]-x,y[\subset U} M^*((x,y))$$
\end{pf}
\begin{remark}
 A corollary of the proof is that the restriction morphism of $\Tilde{M}$ are given, for $U\subset V$, by the canonical restrictions $M^*((a,b)) \to M^*((x,y))$ for any $ U\supset (x,y) \subset (a,b)\subset V$ and the induced (by the universal property) map on the limits. 
\end{remark}
Composing $\tilde{-}$ with the (opposite of the) sheafification functor $ \text{PSh}(\R) \to \Mod(\kk_{\R})$ gives the functor from persistence modules on $\Delta^{+}$ to sheaves on $\R$. 
\begin{defi}\label{D:defofBar}
We set $\bar{M}$ to be the sheaffification of the presheaf $\tilde{M}$ and write 
$\Bar{\cdot} : \Pe(\Delta^+)\longrightarrow \Mod (\kk_\R)^{\op}$ for the induced functor $M\mapsto \bar{M}$. We call $\Bar{\cdot}$ the level-set persistence to sheaves functor. 
\end{defi}
Similarly there is a functor going in the other direction. Indeed, given a sheaf on $\R$, by restriction to open intervals and using the identification of lemme~\ref{L:Defiota}, we get a persistence comodule. Since pointwise duality transforms a persistence comodule into a persistence module we obtain the functor 
\begin{equation}
 \label{eq:defpi}  \pi : \Mod(\kk_\R)^{\op} \longrightarrow \Pe(\Delta^+), \quad F\mapsto \Hom_{\Mod(\kk)}(F_{ | \text{open intervals}};\kk)
\end{equation}
that we call (abusively) the restriction to the intervals functor.

We will also write $\text{bidual}_{\Mod(\kk_\R)}$ the endofunctor of  sheaves  which to a sheaf $M$ 
associate its pointwise bidual $N\mapsto (N^*)^*$. There is a canonical natural transformation 
\begin{equation}\label{eq:unitadjPetoSh} \text{id}_{\Mod(\kk_\R)} \to \text{bidual}_{\Mod(\kk_\R)}\end{equation} given by the pointwise canonical morphism.
\begin{prop}\label{P:PropertiesofBar}
The level-set persistence to sheaves functor $\Bar{\cdot} : \Pe(\Delta^+)\longrightarrow \Mod (\kk_\R)^{\op}$ satisfies the following properties : 
\begin{enumerate}
    \item Its composition  with the restriction to the intervals functors is the canonical biduality functor: $\bar{\cdot}\circ \pi = \text{bidual}_{\Mod(\kk_\R)}$. 
    
    In particular the restriction of this composition of functors to the subcategory of pointwise finite dimensional objects is naturally isomorphic to $\text{id}_{\Mod(\kk_\R)^{pfd}}$
    \item It is the right adjoint of the functor $\pi$: 
    $$\Hom_{\Pe(\Delta^+)}(M, \pi(F)) \; \cong \; \Hom_{\Mod(\kk_\R)^{\op}}(\Bar{M}, F)\; = \; \Hom_{\Mod(\kk_\R)}(F,\Bar{M}) $$ and the map~\eqref{eq:unitadjPetoSh} is the counit of this adjunction.
    \item If $M$ is point-wise finite dimensional, and $M \simeq \oplus_i M_i$, then $\bar{M} \simeq \oplus_i \bar{M_i} $
    \item Assume that $M \in \Obj(\Pe(\Delta^{+}))$ is point-wise finite dimensional, then for all $\alpha \in \R$, we have natural isomorphisms  $$ \varprojlim_{]-x;y[\ni \alpha}M((x,y)) \;\simeq\; \tilde{M}_{\alpha}\; \simeq \; \bar{M}_{\alpha}$$ provided that the left hand side is of finite dimension.
    
    \item One can identify $\Bar{M}$ with the image of the morphism of pre-sheaves : $\Tilde{M}\longrightarrow \prod_{\alpha\in\R}\Tilde{M}_{\alpha} $.
\end{enumerate}

\end{prop}


CAREFUL DUE TO OPPOSITE CATEGORIES AND CHOICE WETHER WE CONSIDER DELTAOP OR MODOP we mau=y ahve TO CHNAGE LEFT TO RIGHT KAN EXTENSIONS AND SO ON. 
\begin{pf}
\begin{enumerate}
    \item For the second part it is sufficient to check that the canonical transformation $\text{id}_{\Mod(\kk_\R)} \to \text{bidual}_{\Mod(\kk_\R)}$ is an isomorphism on all stalks when restricted to a pointwise finite dimensional sheaf, which reduces the statement to the standard case of finite dimensional vector spaces.  For the first part, if $F$ is a sheaf, then its value on an open $U$ is uniquely determined by its value on any open cover and further for any interval $I$ one has that $$\varprojlim_{(-x,y)\subset I} ((F(I))^*)^* \cong (F(I)^*)^*.$$ In particular, on can restrict to cover by open intervals $(I_j)$ of $U$ and compute the value of the sheaf at $U$ as a limit.
   Therefore, noticing that the intersection of two intervvals is an interval, we get that for a sheaf $F$, one has 
   \begin{multline*}\overline{\pi(F)}(U) = \varprojlim \Big(\xymatrix{\overline{\pi(F)}(\coprod U_k) \ar@<1ex>[r] \ar@<-1ex>[r] & \overline{\pi(F)}(\coprod (U_i \cap U_j)) } \Big) \\ 
   \cong \; \varprojlim \Big(\xymatrix{\prod (F(U_k)^*)^* \ar@<1ex>[r] \ar@<-1ex>[r] & \prod(F(U_i \cap U_j)^*)^* } \Big) \\ \cong \; (F(U)^*)^*
   \end{multline*}
   \item Since the sheafification functor is a right adjoint and by universal property of Kan extensions  we have natural isomorphisms:
   \begin{multline*}
  \Hom_{\Mod(\kk_\R)^{\op}}(\Bar{M}, F)  = \Hom_{\Mod(\kk_\R)}(F,\Bar{M})    \\ \cong \;
  \Hom_{\text{PSh}(\R)}(F, \tilde{M}) \; \cong \; \Hom_{\text{PSh}(\R)}(F, \text{Lan}_{\iota^{\op}} M^*) \\ \cong \; \Hom_{\Pe(\Delta^+)}(F_{ | \text{op. intervals} }, \Hom_{\kk}(M, \kk)) \\ \cong \; \Hom_{\Pe(\Delta^+)}(F_{ | \text{op. intervals} }\otimes M, \kk) \\ \cong \; 
  \Hom_{\Pe(\Delta^+)}(M, \Hom_{\kk}(F_{ | \text{op. intervals} }, \kk)) \\ \cong \; 
 \Hom_{\Pe(\Delta^+)}(M, \pi(F)). 
   \end{multline*}
This proves the adjunction formula. The fact that the counit is given by~\eqref{eq:unitadjPetoSh} is 
a direct consequence of the proof and the proof of property (1).
    \item This comes from the facts that $\Hom(-;\kk)$ and sheafification (which is a right adjoint) commutes with finite direct sums, and that left Kan extensions commute to arbitrary direct sums.
    \item Write $\text{Int}(\alpha)$ for the (full) subcategory of $\Ouv(\R)$ consisting of intervals containing $\alpha$. Let us fix  $G :  (]0,\infty[,\leq)  \longrightarrow \text{Int}(\alpha)$ defined by 
  $G(\varepsilon) = ]\alpha - \varepsilon , \alpha + \varepsilon[  $. Then $G$ is a functor and is initial among functors $ (]0,\infty[,\leq)  \longrightarrow \text{Int}(\alpha)$. Therefore : 
     $$ \varprojlim_{]-x;y[\ni \alpha}M((x,y)) \simeq \varprojlim_{\varepsilon > 0}M((\varepsilon - \alpha  , \alpha + \varepsilon)) $$
     
     
     Since $(]0,\infty[,\leq) $ is a totally ordered set, we can apply the theorem of decomposition of pfd modules over totally ordered sets to $M\circ G$, thus there exists a multiset $\mathbb{B}(M\circ G)$ of intervals of $\R$ such that : 
     \begin{equation}\label{eq:DefMoG} M\circ G \simeq \bigoplus_{I\in \mathbb{B}(M\circ G)} \kk_I \end{equation}
    It follows that \begin{equation}\label{eq:porjlimM} \varprojlim_{\varepsilon > 0}M((\varepsilon - \alpha  , \alpha + \varepsilon)) \simeq \prod_{\substack{I\in \mathbb{B}(M\circ G)\\ 0 \in \text{closure}(I)}} \kk  \end{equation}
                                                                                                                                                                                                                  
     
     Now if $\varprojlim\limits_{]-x;y[\ni \alpha}M((x,y))$ is finite dimensional then the above product in the right hand side of~\eqref{eq:porjlimM} is a finite product and thus a direct sum: 
     $\prod_{\substack{I\in \mathbb{B}(M\circ G)\\ 0 \in \text{closure}(I)}} \kk \; \simeq \; \bigoplus_{\substack{I\in \mathbb{B}(M\circ G)\\ 0 \in \text{closure}(I)}}.$ Therefore we have 
     \begin{align*}
        \varprojlim_{]-x;y[\ni \alpha}M((x,y))% &\simeq\prod_{\substack{I\in \mathbb{B}(M\circ G)\\ 0 \in \text{closure}(I)}} \kk  \\
        & \simeq \bigoplus_{\substack{I\in \mathbb{B}(M\circ G)\\ 0 \in \text{closure}(I)}} \kk  \\
        &\simeq \varprojlim_{]-x;y[\ni \alpha} \Hom \left ( M((x,y)), \kk \right ) \; \text{ (by~\eqref{eq:DefMoG} and finite dimensionality)}\\ 
        &\simeq \tilde{M}_\alpha \; \text{ (by~\eqref{eq:defMtilde})}\\
        &\simeq \bar{M}_\alpha.
    \end{align*}
    \item This is a general fact for sheaves on a $T_1$-topological space, that is a space for which all   points are closed.
\end{enumerate}
\end{pf}

Let $\Delta = \{(-x,x)\mid x\in\R\}$, and $p : \Delta \longrightarrow \R$ the projection onto the second coordinate. Recall that for any block $B$ (Definition~\ref{def:block_MV}) we have defined (see~\ref{Def:blockmodule}) a persistence module $\kk^B\in \Obj(\Pe(\Delta^+))$.
\begin{prop}\label{P:BarofBlock}
Let $B$ be a block. Let $a,b\in \R$ such that $<a,b> = p(B\cap \Delta)$, with the convention that $a=1$ and $b=-1$ when $ p(B\cap \Delta) = \emptyset$.  
\begin{enumerate}
    \item If $B$ is of type \textbf{dquad}, then $\overline{\kk^B} \simeq \kk_{]a,b[}$.
    \item If $B$ is of type \textbf{bquad}, then $\overline{\kk^B} \simeq \kk_{[a,b]}$.
    \item If $B$ is of type \textbf{vquad}, then $\overline{\kk^B} \simeq \kk_{]a,b]}$.
    \item If $B$ is of type \textbf{hquad}, then $\overline{\kk^B} \simeq \kk_{[a,b[}$.
\end{enumerate}
\end{prop}
\begin{remark}
In particular, $\bar{\kk^B}$ does not depend on whether $B$ contains its boundary or not and, if $B$ is of type $\textbf{bb}^+$, then $\overline{\kk^B} =0$. 
\end{remark}
\begin{remark}\label{R:caracofab}
 If $B$ is of type $\textbf{bb}^-$, then the numbers $a$ and $b$ satisfies that the point $(-a,b)$ is the infimum of the points in $B$.
 
 Similarly, if $B$ is of type $\textbf{dquad}$, then the numbers $a$ and $b$ satisfies that the point $(-a,b)$ is the supremum of the points in $B$.
\end{remark}


\begin{cor}\label{C:pfdimpliesconstructible}
 If $M \in \Obj(\Pe(\Delta^{+}))$ is pointwise finite dimensional, then $\Bar{M}$ is weakly constructible. 
 Further if $M$ is strongly pointwise finite dimensional (definition~\ref{D:spfd}), then $\Bar{M}$ is constructible. 
\end{cor}
In particular, the restriction of the sheafification functor $\bar{\cdot}:\Pe(\Delta^{+})\to \Mod(\kk_\R)$ to the full subcategory of pfd modules takes values in the subcategory $ \Mod_{\R c}(\kk_\R)$ of constructible sheaves.
\begin{pf}
 By the decomposition Theorem~\ref{thm:exactdecomp}, the pfd module $M$ is isomorphic to a direct sum of blocks $M\cong  \bigoplus_{B\in \mathbb{B}(M)} \kk^B$. Since $\Bar{\cdot}$ commutes with direct sum for pfd modules (Proposition~\ref{P:PropertiesofBar}), Proposition~\ref{P:BarofBlock} yields that $\overline{M} \cong \bigoplus_{B\in \mathbb{B}(M)} \overline{\kk^B} $ is a (pointwise finite when $M$ is strongly pfd) direct sum of sheaves of the form $\kk_{I}$ where $I$ is an interval. 
\end{pf}


The level set persistence to sheaves functor $\Bar{(\cdot)}$ does not preserve
interleavings in general. However, the trouble is only related to the death or $\textbf{bb}^+$ quadrant. More precisely we have the following two lemmas.
\begin{lem}\label{L:Barpreserveentre} Let $M, N\in \Obj(\Pe(\Delta^+))$ be pointwise finite dimensional and such that their barcodes contains only blocks of type $\textbf{bb}^{-}$, $\textbf{vb}$ and $\textbf{hb}$. 
Then $$\overline{M [\vec{\varepsilon}]} \cong \Bar{M}\star K_{\varepsilon}. $$ Further, 
if $M \sim_\varepsilon^{\Delta^+} N$, then 
 $$ \Bar{M} \; \sim_\varepsilon \, \Bar{N}.$$
\end{lem}
\begin{pf}
 By Theorem~\ref{thm:exactdecomp}, we have  isomorphisms 
 $M \cong\bigoplus_{B\in \mathbb{B}(M)} \kk^B$, 
 $N \cong \bigoplus_{B\in \mathbb{B}(N)} \kk^B$ of persistence modules, such that the blocks $B$ are $\textbf{bb}^{-}$, $\textbf{vb}$ and $\textbf{hb}$. Lemma~\ref{L:shiftofBlock} 
 implies that 
 \begin{align*}\overline{M[\vec{\varepsilon}]} &\cong \bigoplus_{B\in \mathbb{B}(M)} \overline{\kk^{B -\vec{\varepsilon}}}  \\ 
  \quad \overline{N[\vec{\varepsilon}]} &\cong \bigoplus_{B'\in \mathbb{B}(N)} \overline{\kk^{B' -\vec{\varepsilon}} }
  \end{align*}
 where each $\overline{\kk^{B -\vec{\varepsilon}}}$ is of the form 
 $\kk_{I(B, \varepsilon)}$ where $I(B, \varepsilon)$ is an interval $<a,b>=p((B-\vec{\varepsilon})\cap \Delta)$ which is
 \begin{itemize}\item a closed non-empty interval if $B$ is of type $\textbf{bb}^{-}$ depends only on the type (death, birth, vertical...) of $B$; 
  \item a semi-open interval closed on the left (resp. closed on the right) if  $B$ is of type $\textbf{hb}$ (resp. $\textbf{vb}$.
 \end{itemize}
Therefore we have: 
\begin{eqnarray*} 
 \text{if $B$ is of type $\textbf{bb}^{-}$, then }  \overline{\kk^{B-\vec{\varepsilon}}} 
 & \cong & \kk_{[a-\varepsilon, b+\varepsilon ]}, \\
 \text{if $B$ is of type \textbf{vquad}, then }  \overline{\kk^{B-\vec{\varepsilon}}} 
 & \cong & \kk_{]a+\varepsilon, b+\varepsilon ]}, \\
 \text{if $B$ is of type \textbf{hquad}, then }  \overline{\kk^{B-\vec{\varepsilon}}} 
 & \cong & \kk_{[a-\varepsilon, b-\varepsilon [}.
\end{eqnarray*}
 Using~\cite[Proposition 3.8]{Berk18}, we thus get that  in all cases, 
 $$ \overline{\kk^{B-\vec{\varepsilon}}} 
 \; \cong \;  \overline{\kk^B} \star K_\varepsilon $$ and by additivity of the convolution functor we obtain $\overline{M [\vec{\varepsilon}]} \cong \Bar{M}\star K_{\varepsilon}$ as claimed.
 
 \smallskip
 
 The same results holds for  the blocks $B'\in 
 \mathbb{B}(N)$ so that 
 $$\overline{M [\vec{\varepsilon}]} \cong \Bar{M}\star K_{\varepsilon}, \quad  \overline{N [\vec{\varepsilon}]} \cong \Bar{N}\star K_{\varepsilon}$$
 Now let $f: M\to N[\vec{\varepsilon}]$ and $g: N\to M[\vec{\varepsilon}]$ be an $\varepsilon$-interleaving between $M$ and $N$, then applying the functor $\Bar{\cdot}$ to the latter isomorphisms, we obtain  an $\varepsilon$-interleaving in sheaves given by  
 $ \Bar{M}\stackrel{\Bar{f}}\to \overline{N[\vec{\varepsilon}]}\cong \overline{N}\star K_\varepsilon$ 
 and $\Bar{N}\stackrel{\Bar{g}}\to \overline{M[\vec{\varepsilon}]}\cong \overline{M}\star K_{\varepsilon}$.
\end{pf}

For \emph{deathblocks} or \emph{birthblocks} of type $\textbf{bb}^{+}$, the sheafification functor does not intertwine shifts with convolution in a naive way. 
However we have the following precise result for which we first 
recall that to a block $B$, we can associate the two real numbers $a,b\in \R$ such that $<a,b> = p(B\cap \Delta)$;  the convention being that $a=1$ and $b=-1$ when $ p(B\cap \Delta) = \emptyset$. 
If $B$ is a $\textbf{bb}^{+}$ block, its dual death block $B^\dagger$ intersects $\Delta$ and we denote $<a^\dagger,b^\dagger> = p(B^\dagger \cap \Delta)$.
\begin{lem}\label{L:Barondeathblocks}Let $B$ be a block of type $\textbf{db}$ or $\textbf{bb}^{+}$.  Then for $\varepsilon \geq 0$,
 \begin{eqnarray}
 \text{if $B$ is of type \textbf{dquad}, then, } \overline{\kk^{B}}[\vec{\varepsilon}] 
 & \cong & \left\{ \begin{array}{ll}  \overline{\kk^{B}}\star K_{\varepsilon} & \mbox{ if $\varepsilon< \frac{b-a}{2}$}\\
 0 & \mbox{ if $\varepsilon \geqslant \frac{b-a}{2}$,} \end{array}                                                                                                                          \right .  \label{eq:Barofdb}\\
 \text{if $B$ is of type $\textbf{bb}^{+}$, then, }\overline{\kk^{B}}[\vec{\varepsilon}] 
 & \cong &   \left\{ \begin{array}{ll}  0 & \mbox{ if $\varepsilon< \frac{b^\dagger-a^\dagger}{2}$}\\
 \kk_{[\frac{a^\dagger+b^\dagger}{2}, \frac{a^\dagger+b^\dagger}{2}]}\star K_{\varepsilon -\frac{b^\dagger -a^\dagger}{2}} & \mbox{ if $\varepsilon \geqslant \frac{b^\dagger-a^\dagger}{2}$.} \end{array}                                                                                                                          \right .  \label{eq:Barofbb+}
 \end{eqnarray}
\end{lem}
\begin{pf}The proof will be similar to the one of lemma~\ref{L:Barpreserveentre}. First note that,  if $B$ is of birthtype $\textbf{bb}^+$, the supremum of $B^\dagger$ is the infimum of $B$ by definition of the dual block. Therefore, by remark~\ref{R:caracofab}, we have that the infimum of the elements of $B$ is the point $(-a^\dagger, b^\dagger)\in \Delta^+$. 
It follows that $B-\vec{\varepsilon}$ remains of type $\textbf{bb}^+$ as long as $\varepsilon < \frac{b^\dagger-a^\dagger}{2}$ and it becomes of type $\textbf{bb}^{-}$ when  $\varepsilon \geqslant \frac{b^\dagger-a^\dagger}{2}$. Further, in that latter case, we have that 
$$ p((B-\vec{\varepsilon})\cap \Delta) = \langle b^\dagger  -\varepsilon,  a^\dagger + \varepsilon \rangle.  $$
By lemma~\ref{L:shiftofBlock} and proposition~\ref{P:BarofBlock}, we thus have that {if $B$ is of type $\textbf{bb}^{+}$, then }
 \[
   \overline{\kk^{B-\vec{\varepsilon}}} 
 \, \cong \, \left\{ \begin{array}{ll}  0 & \mbox{ if $\varepsilon< \frac{b^\dagger-a^\dagger}{2}$}\\
 \kk_{[b^\dagger  -\varepsilon,  a^\dagger + \varepsilon]}& \mbox{ if $\varepsilon \geqslant \frac{b^\dagger-a^\dagger}{2}$.} \end{array}                                                                                                                          \right . \]
 Using~\cite[Proposition 3.8]{Berk18}, we see that for  $\varepsilon \geqslant \frac{b^\dagger-a^\dagger}{2}$, one has $$\kk_{[b^\dagger  -\varepsilon,  a^\dagger + \varepsilon]} 
 \, \cong \,\kk_{[\frac{a^\dagger+b^\dagger}{2}, \frac{a^\dagger+b^\dagger}{2}]}\star K_{\varepsilon -\frac{b^\dagger -a^\dagger}{2}} $$
 which shows the formula~\eqref{eq:Barofbb+}. 
 
 Now, note that if $B$ is of type \textbf{dquad}, then $B-\vec{\varepsilon}$ has a non-empty intersection with $\Delta$ as long as $\varepsilon < \cfrac{b-a}{2}$. And similarly we find, using proposition~\ref{P:BarofBlock} that 
 
 \[
  \overline{\kk^{B-\vec{\varepsilon}}} 
 \, \cong \,\left\{ \begin{array}{ll}  0 & \mbox{ if $\varepsilon \geq \frac{b^\dagger-a^\dagger}{2}$} \\  \kk_{]a+\varepsilon, b-\varepsilon [}  & \mbox{ if $\varepsilon < \frac{b^\dagger-a^\dagger}{2}$}.\end{array}                                                                                                                          \right . \]
 To prove formula~\eqref{eq:Barofdb}, we are left to apply ~\cite[Proposition 3.8]{Berk18} a last time.
\end{pf}


IMPORTANT: FAUT IL ECRIRE $\Delta=\{ (x,-x)\}$ PLUTOT ? IL ME SMEBLE QUE CA REPRESENTE MIEUX LA FACON DONT ON PARAMETRE L'ANTI DAIOGNALE (SUIVANT les $x$ CROISSANT)

\emph{
Lien : l'extension sur la diagonale représente la faisceautisation
Au passage : decomposition des pré-faisceaux de MV qui sont de dim finie sur les ouvert relativement compacts
}


\section{ Almost isometric equivalence between $\mbox{MV}(\R)^\s$ and $\D^b_{\R c}(\kk_\R)$}
In this section, we explain why the interleaving distance between level set persistence is essentially the same as the derived bottleneck distance between the associated sheaves.

In order to express this we will relate constructible sheaves by an isometry to a specific type of graded persistence modules, that is those satisfying the following definition.
\begin{defi}\label{D:spfd}
 A middle-exact persistence module $M\in \Pers(\Delta^+)$, is said to be \textbf{strongly pointwise finite dimensional}, if it is pointwise finite dimensional and satisfies the following additional condition : 

$$\text{For every $\alpha \in \Delta$,}~ \varprojlim_{]-x;y[\ni \alpha}M((x,y))  ~ \text{is finite dimensional} $$

A  Mayer-Vietoris system $S=(S_i,\delta^s_i)$ is said to be strongly pointwise finite dimensional if each $S_i$ is strongly pointwise finite dimensional  and only finitely many  $S_i$'s are non-zero.

The full subcategory of MV($\R$) whose objects are strongly pointwise finite dimensional MV-systems is denoted by $\text{MV}(\R)^\s$.
\end{defi}



We will now  build two functors : 


$$\xymatrix{
(\overline{~\cdot~ })^{\text{MV}} & : \text{MV}(\R)^\s \ar[r] &\D^b_{\R c}(\kk_\R)  \\
\pi^{\text{MV}}  & : \D^b_{\R c}(\kk_\R) \ar[r] & \text{MV}(\R)^\s 
}$$

Satisfying $(\overline{~\cdot~ })^{\text{MV}} \circ \pi^{\text{MV}} \simeq \text{id}_{\D^b_{\R c}(\kk_\R)}$ and that for every $M\in \text{MV}(\R)^\s$, $$d_I^{\text{MV}}(M, \pi^\MV ( \overline{M})) = 0.$$
See Corollary~\ref{C:MVcircPsiisId} and Corollary~\ref{C:MVdisO}.
\subsection{Construction of $(\overline{~\cdot~ })^{\text{MV}}$}
We will now apply section~\ref{SS:levelsettopresheaves} to compare the Mayer-Vietoris persistence systems and constructible sheaves. To do so, we first consider the direct sum of the level set persistence to sheaves functor: 

Let $\ell$ be the localization functor sending the category of complexes of sheaves over $\R$ to its derived category $\D(\kk_\R)$.
\begin{defi}\label{D:MVtoSheaves} The \emph{M-V sheafification functor}: $\overline{(-)}^{MV}: \text{M-V}(\R) \to  \D(\kk_\R)$ is the functor given, on
objects $S=(S_i, \delta_i^S)_{i\in \Z, s\in \R^2_{>0}}\in \Obj(\text{M-V}(\R))$, by 
$$ \overline{S}^{MV} := \ell\Big(\bigoplus_{i\in \Z} \overline{S_i}[i]\Big)$$ and, on morphisms $(S_i\stackrel{\varphi_i}\to T_i)_{i\in \Z}$, by 
$$\overline{(\varphi_i)_{i\in \Z}} := \ell\left(\bigoplus \overline{\varphi_i}\right). $$ 
\end{defi}
That this is a functor is a direct consequence of section~\ref{SS:levelsettopresheaves}.

\begin{lem}
 If $S$ is a pointwise finite dimensional Mayer-Vietoris system, then $\overline{S}^{MV}$ is a constructible sheaf. In particular, we have a commutative diagram of functors:
 $$\xymatrix{ \text{M-V}(\R) \ar[r]^{\quad\overline{(-)}^{MV}} & \D(\kk_\R) \\ 
 \text{M-V}(\R)^{\s} \ar@{^{(}->}[u]\ar[r]^{\quad\overline{(-)}^{MV}} & \D^b_{\R c}(\kk_\R) \ar@{^{(}->}[u].}  $$
\end{lem}
\begin{pf}
 We can apply Theorem~\ref{thm:decom_MV} together with proposition~\ref{P:BarofBlock} in a way similar to the proof of corollary~\ref{C:pfdimpliesconstructible}.
\end{pf}



\begin{prop}\label{P:PropertiesofMVSheafification}
The M-V sheafification functor $\overline{(-)}^{MV}: \text{M-V}(\R)^{\s} \to  \D^b_{\R c}(\kk_\R)$ satisfies the following properties : 
\begin{enumerate}
    \item it commutes with the shift in gradation: for all Mayer-Vietoris system $M$, one has $\overline{M[n]}^{MV} \cong \overline{M}^{MV}[n]$. 
    \item For $B$ a block of type \textbf{bb$^-$}, \textbf{hb}, \textbf{vb}, \textbf{db}, $j\in \Z$ and $\varepsilon \geq 0$, we have : $$(\overline{S_j^B [\varepsilon]})^{\MV} \simeq \kk_B[-j] \star K_\varepsilon $$
    \item if $\overline{M}^{MV}$ is isomorphic to  $\overline{N}^{MV}$ (in the derived category), then $d_I^{MV}(M, N)) =0$.
 \end{enumerate}
\end{prop}
\begin{pf}
First note that assertion 1 is immediate from the definition since we put each $S_i$ in degree $i$.   


\smallskip 

2. First assume $B$ is of type  \textbf{bb$^-$}, \textbf{hb} or \textbf{vb}. Then  definition~\ref{D:blocksmodulesforMV} implies that $S_j^B \cong \kk_B[-j]$. Since  $\overline{(\cdot)}^{MV}$ commutes with direct 
sum and shifts, Lemma~\ref{L:Barpreserveentre} implies $(\overline{S_j^B [\varepsilon]})^{\MV} \simeq \kk_B[-j] \star K_\varepsilon $. It remains to consider the case of a block of type \textbf{db}. Then definition~\ref{D:blocksmodulesforMV} says that as a graded persistent module, one has $$S_j^B \cong \kk_B[-j] \oplus \kk_{B^\dagger}[-j-1]$$ and therefore  $$\overline{S_j^B[\vec{\varepsilon}]}^{MV} \cong \overline{\kk_B[\varepsilon]}[-j] \oplus \overline{\kk_{B^\dagger}[\varepsilon]}[-j-1].$$ Denote $\langle a, b\rangle =p(B \cap \Delta)$ as before Proposition~\ref{P:BarofBlock}. Following the notation of 
Lemma~\ref{L:Barondeathblocks} we thus have that for the dual block $B^\dagger$ of type \textbf{bb}$^{+}$, one has that $a^\dagger=a$, $b^\dagger=b$ by definition. Then Lemma~\ref{L:Barondeathblocks}, commutation of convolution wih shifts and Proposition~\ref{P:BarofBlock} imply that 
\begin{multline}\label{eq:MVonSofDB}
\overline{\kk_B[\varepsilon]}[-j] \oplus \overline{\kk_{B^\dagger}[\varepsilon]}[-j-1] \;
  \cong \;\left\{ \begin{array}{ll}  \kk_{]a,b[}\star K_{\varepsilon}[-j] & \mbox{ if $\varepsilon< \frac{b-a}{2}$}\\
 \kk_{[\frac{a+b}{2}, \frac{a+b}{2}]}\star K_{\varepsilon -\frac{b -a}{2}} [-j-1] & \mbox{ if $\varepsilon \geqslant \frac{b-a}{2}$.} \end{array}   \right .
\end{multline}
This formula ~\eqref{eq:MVonSofDB} is precisely the formula for $\kk_{]a,b[}\star K_{\varepsilon}$ according to~\cite[Proposition 3.8]{Berk18}. This concludes the proof of claim 2.

\smallskip

3. Assume $\overline{M}^{MV} \cong \overline{N}^{MV}$. By Theorem~\ref{thm:decom_MV}, we decompose 
$M\cong \simeq \bigoplus_{j\in \Z} \bigoplus_{B_M \in \mathbb{B}_j(M)}S_j^{B_M}[j]$ and 
$N \cong \simeq \bigoplus_{j\in \Z} \bigoplus_{B_N \in \mathbb{B}_j(N)}S_j^{B_N}[j]$ into  Mayer-Vietoris blocks.
Since $\overline{(\cdot)}^{MV}$ commutes with direct 
sum and shifts (by property 1),  we have isomorphisms
\begin{eqnarray*}
 \overline{ \bigoplus_{j\in \Z} \bigoplus_{B_M\in \mathbb{B}_j(M)}S_j^{B_M}[j]}^{MV} &\cong&
 \overline{\bigoplus_{j\in \Z} \bigoplus_{B_N\in \mathbb{B}_j(N)}S_j^{B_N}[j] }^{MV}  \\ 
 \bigoplus_{j\in \Z} \bigoplus_{B_M\in \mathbb{B}_j(M)} \overline{S_j^{B_M}}^{MV}[j] &\cong& 
 \bigoplus_{j\in \Z} \bigoplus_{B_N\in \mathbb{B}_j(N)} \overline{S_j^{B_N} }^{MV}[j].
\end{eqnarray*}
For any vertical, horizontal or \textbf{bb}$^-$ type block $B$, Proposition~\ref{P:BarofBlock} tells us that  
$\overline{S_j^{B}}^{MV} \cong \kk_{I(B)}$ where $I(B)$ is a non-empty interval (uniquely determined by $p(B\cap \Delta)$. 
If $B$ is of type \textbf{db}, then 
$$
 \overline{S_j^{B}} \; \cong \;  \kk_{I(B)} \oplus \kk_{I(B^\dag)}[1]
$$
according to definition~\ref{D:blocksmodulesforMV} and~\ref{D:MVtoSheaves}.  
Therefore, we have an isomomorphism 
\begin{multline}
 \label{eq:decofBarMV} 
 \bigoplus_{j\in \Z} \left(\Big(\bigoplus_{B_M\in \mathbb{B}_j(M)\setminus \mathbb{B}^{\textbf{dq}}_j(M)} \kk_{I(B_M)}[j] \Big) \oplus 
 \Big(\bigoplus_{B_M\in \mathbb{B}^{\textbf{dq}}_j(M) } \big(\kk_{I(B_M)}[j] \oplus \kk_{I(B_M^\dag)}[j+1]\Big)\right)\\
 \cong \; 
 \bigoplus_{j\in \Z} \left(\Big(\bigoplus_{B_N\in \mathbb{B}_j(N) \setminus \mathbb{B}^{\textbf{dq}}_j(N)} \kk_{I(B_N)}[j]\Big)\oplus 
\Big( \bigoplus_{B_N\in \mathbb{B}^{\textbf{dq}}_j(N) } \big(\kk_{I(B_N)}[j] \oplus \kk_{I(B_N^\dag)}[j+1]\Big)\right).
\end{multline}
of constructible sheaves. By unicity of the decomposition in Theorem~\ref{T:KSdecomposition}, we obtain  degreewise bijections between 
the set of associated graded barcodes $\{ I(B_M), \, B_M \in  \mathbb{B}_j(M)\}$ and $\{ I(B_N), \, B_N \in  \mathbb{B}_j(N)\}$
and therefore  bijections $\sigma_j: \mathbb{B}_j(M) \cong \mathbb{B}_j(N)$ with the property that  for any $B_M\in \mathbb{B}_j(M)$, 
$\sigma_j(B_M)$ is a block of the same type as $B_M$ and which is equal to $B_M$ except maybe on the boundary.



\begin{lem} Let $\mathcal{B}, \mathcal{B}'$ be sets of M-V blocks of types \textbf{db}, \textbf{vb}, \textbf{db} and \textbf{bb}$^-$. 
If $\sigma: \mathcal{B} \to \mathcal{B}'$ is a bijection such that for any $B\in \mathcal{B}$, $\sigma(B)$ is equal to $B$ except maybe on the boundary, then 
$$d_I^{MV}\left(\bigoplus_{B\in \mathcal{B}} S^B_{j}, \bigoplus_{B'\in \mathcal{B}'} S^{B'}_{j}\right) =0.$$
\end{lem}
\begin{pf}[Proof of the lemma] It is enough to check that,
 if $B$ and $B'$ are two blocks of the same type which differs only on their boundary, then $B$ and $B'$ are $\varepsilon$-interleaved for any $\varepsilon >0$. This property follows from Lemma~\ref{L:shiftofBlock} and an immediate application of the definition of the blocks of each type. 
 Then  the direct sum of those interleavings relating each $\mathcal{B}$ to $\sigma(\mathcal{B})$  gives a $\varepsilon$-interleaving in between $\bigoplus_{B\in \mathcal{B}} S^B_{j_B}$ and  $\bigoplus_{B'\in \mathcal{B}'} S^{B'}_{j_{B'}}$ for every $\varepsilon >0$; the lemma follows.
\end{pf}
The lemma gives the claimed property 3 since we just prove that we can find such a 
permutation relating $\mathbb{B}_j(M)$, $\mathbb{B}_j(N)$ for each degree $j$.
\end{pf}



\subsection{Construction of $\Psi$}

Given $F \in \D^b_{\R c}(\kk_R)$ and $i \in \Z$, define $\Psi(F)_i$ to be the object of $\Pers(\Delta^+)$ defined by, for $(x,y)\in \Delta^+$ : 

$$\Psi(F)_i =\bigoplus_{k+l = i} \Hom_{\Mod(\kk)}\left ( \Rr^k\Gamma \left ( (-x,y) , \Sigma^l F\right ), \kk \right )  $$

Where $\Rr^k\Gamma \left ( (-x,y) , -\right )$ is the $k$-th right derived functor of the functor of sections on $(-x,y)$, in other words the $k$th cohomology groups of $\Sigma^l F_{|(-xy)}$. Note that since $F$ is assumed to be constructible, there are only finitely many pairs $(k,l)$ such that the right-hand-side vector space is non zero.

\begin{prop}\label{P:psiofCOnstrisMV}
The family $(\Psi(F)_i)_{i\in \Z}$ carries the structure of a Mayer-Vietoris system. In addition, it is strongly pointwise finite dimensional. 
\end{prop}

\begin{pf} The fact that $\Psi_i(F)$ is a persistence module over $\Delta^+$ is an immediate consequence of lemma~\ref{L:Defiota}.
For $s = (s_1,s_2)\in \R^2_{>0}$ and $i\in\Z$, we have to build the connection morphism $\delta_i^2$. Let $I^\bullet\in C^b(\kk_\R)$ an injective resolution of $F$ in the category of constructible sheaves. Consider $(-x,y)\in \Delta^+$, then we have the Mayer-Vietoris sequence associated to the cover $(-x - s_1 , y) \cup (-x , y + s_2) $ of $(-x - s_1 , y + s_2)$ which is the short exact sequence of complexes of sheaves 

$$0 \longrightarrow \Gamma((-x - s_1 , y + s_2), I^\bullet) \longrightarrow \Gamma((-x - s_1 , y), I^\bullet) \oplus  \Gamma((-x , y+s_2), I^\bullet) \longrightarrow \Gamma((-x,y), I^\bullet) \longrightarrow 0 .$$
Passing to cohomology, we obtain a long exact sequence (see~\cite{Kash90}) 
\begin{multline}\label{eq:lessheavesforPi} 
\dots \to     H^i((-x-s_1, y+s_2, I^\bullet) \to 
 H^i((-x-s_1, y, I^\bullet) \oplus  H^i((-x, y+s_2, I^\bullet) \\ \to 
  H^i((-x, y, I^\bullet) \stackrel{\delta}\to H^{i+1}((-x-s_1, y+s_2, I^\bullet) \to \dots.
\end{multline}
Since by definition of sheaf cohomology, one has, $H^i((-x, y, I^\bullet) \cong \Rr^i\Gamma \left ( (-x,y) , \Sigma^l F\right )$ and $\Sigma^l I^\bullet$ is an injective resolution of $\Sigma^l F$, the linear dual of the direct sum of the maps $\delta$ given by the exact sequence~\eqref{eq:lessheavesforPi} yields linear maps $\delta_i^s:\psi_i(F)((x,y)) \to \psi(F)[s](x,y)$ for all $(x,y) \in \Delta^+$. The exactness of~\eqref{eq:lessheavesforPi} and Lemma~\ref{L:Defiota} also implies that the collection $(\psi_i(F), \delta_i^s)_{i,s}$ is a Mayer-Vietoris system over $\R$.

Since $F$ is constructible, its cohomology groups are finite dimensional in each degree, and there are only finitely many of them. Therefore $\Psi(F)$ is pointwise finite dimensional. Now the proof that $\Psi(F)$ is strongly finite dimensional is an argument similar to the proof of property 4 in Proposition~\ref{P:PropertiesofBar}. Alternatively, one can simply use the structure theorem~\ref{T:KSstructure}
and proposition~\ref{P:PsionIntervals} below to conlude directly since  strongly pointwise finite dimensional modules are stable under locally finite direct sums.\end{pf}

\begin{cor}\label{P:PropertiesofPsi}
The rule $F\mapsto \Psi(F)$ defines a functor $\Psi: \D^b_{\R c}(\kk_R)\to \MV(\R)^\s$ 
which is additive and commutes with shifts.
\end{cor}
\begin{pf} Since the definition of $\Psi_i$ and the connecting morphism in Mayer-Vietoris long exact sequences are functorial, the fact that $\Psi= (\Psi_i(F), \delta^s_i)_{i,s}$ is 
The first part is a consequence of the fact that the shift commutes with direct sum as well as the fact that the cohomology functor commutes with direct sum and shifts. 
\end{pf}
Recall the definition of the map $p: \Delta\to \R$ given before Proposition~\ref{P:BarofBlock}.
\begin{lem}\label{L:Blockassociatedtointerval}
 Let  $a<b$ be real numbers. There are unique  blocks $B_{b}^{\langle a, b\rangle}$, $B_{h}^{\langle a, b\rangle}$, $B_{v}^{\langle a, b\rangle}$ and $B_{d}^{\langle a, b\rangle}$ respectively of type $\textbf{bb}^{-}$, $\textbf{hb}$, $\textbf{vb}$ and $\textbf{db}$ such that $p(\Delta\cap B_{-}^{\langle a, b\rangle}) = \langle a,b \rangle$. 
\end{lem}
\begin{pf}
By definition~\ref{def:block_MV}, all the blocks except the birth blocks lying entirely in $\Delta^+_{>0}$, that is those of type $\textbf{bb}^{+}$  are uniquely determined by their intersection with the diagonal. RENVOYER A UNE FIGURE
\end{pf}


Now recall the definition~\ref{D:blocksmodulesforMV} of the canonical MV-systems associated to blocks.
\begin{prop}\label{P:PsionIntervals}  Let $I=\langle a, b\rangle$ be an interval in $\R$. 
\begin{enumerate}
    \item If $I$ is open  $\Psi ({\Sigma^i \kk_{\langle a, b\rangle}}) \cong 
S_i^{B_{d}^{\langle a, b\rangle}}$.
    \item If $I=]a,b]$, then  $\Psi ({\Sigma^i \kk_{\langle a, b\rangle}}) \cong 
S_i^{B_{v}^{\langle a, b\rangle}}$.
    \item If $I=[a, b[$, then  $\Psi ({\Sigma^i \kk_{[ a, b[}}) \cong 
S_i^{B_{h}^{\langle a, b\rangle}}$. 
    \item If $I$ is compact, then  $\Psi ({\Sigma^i \kk_{\langle a, b\rangle}}) \cong 
S_i^{B_{b}^{\langle a, b\rangle}}$. 
\end{enumerate}
Here all the isomorphisms are isomorphisms of Mayer-Vietoris systems.
\end{prop}
\begin{pf}Let us first prove the open interval case.
In view of the proof of proposition~\ref{P:psiofCOnstrisMV}, using compatibility with shifts and direct sums, we only need to compute the cohomology groups of $\Rr^k\Gamma \left ( (-x,y) , \kk_{I}\right )$ which by definition is isomorphic to 
$Ext^k_{\kk_\R}\left( \kk_{]-x,y[}, \kk_{I }\right) $. 
By Proposition 3.13 and 3.14 in~\cite{Berk18}, we have that it is always $0$ for $k>1$. Further, the only case for which it is non-zero for $k=1$ is when $I$ is an open and is included in $]-x,y[$ and we then have $Ext^1_{\kk_\R}\left( \kk_{]-x,y[}, \kk_{I }\right) \cong \kk $.  Therefore, by functoriality of the $Ext^1_{\kk_\R}( -, \kk_I)$ functor in its left variable, it follows that the persistence module associated to  
$Ext^1_{\kk_\R}( -, \kk_I)$ in $\Psi(\kk_I)$ is either $0$ if $I$ is not open or, if $I$ is open, is precisely the block module $\Sigma \kk^{(B_d^{I})\dag}$ in degree 1 supported on the type $\textbf{bb}^{+}$ block dual to the deathblock $B_d^{I}$. Here, by block module we refer to Definition~\ref{Def:blockmodule}.

It remains to compute the image of the $Ext^0_{\kk_\R}\left( \kk_{]-x,y[}, \kk_{I }\right) $. By Proposition 3.13 and 3.1 in~\cite{Berk18}, we find that if $I$ is open, 
$$Ext^0_{\kk_\R}\left( \kk_{]-x,y[}, \kk_{I }\right) \cong \left \{ \begin{array}{ll}
\kk & \mbox{if } ]-x, y[ \subset I \\ 0 & \mbox{else. }
\end{array}\right .$$
Using functoriality of $Ext$ again, we thus find that, when $I$ is open, the persistence module associated to $\kk_I$ is the block module $ \kk^{B_d^I}$ concetrated in degree $0$ and supported on the type $\textbf{db}$ block $B_d^I$. Combining the degree $0$ and $1$ part, 
the functoriality of the the Mayer-Vietoris long exact sequence~\eqref{eq:lessheavesforPi} then shows that $\Psi(\kk_I) $ is precisely the MV-block module $S_0^{B_d^I}$ (of Definition~\ref{def:block_MV}). 

Now for the three other types of intervals, the computation is easier since we only have to consider $Ext^0_{\kk_\R}\left( \kk_{]-x,y[}, \kk_{I }\right)$ in the computation of $\Psi(\kk_I)$ (all other degrees are $0$ by the $Ext$ computations of~\cite{Berk18}). Arguing as for the open interval case, Proposition 3.13 and 3.1 in~\cite{Berk18} shows that the persistence modules $Ext^0_{\kk_\R}\left( \kk_{-}, \kk_{I }\right)$ are respectively the block modules $S_0^{B_v^I}$, $S_0^{B_h^I}$ and 
$S_0^{B_b^I}$ when $I$ is of the type $]a, b]$, $[a, b[$ or compact. 
\end{pf}

\begin{lem}Let $F \in \D^b_{\R c}(\kk_\R)$. For any $\varepsilon \geq 0$, there is a natural isomorphism 
$ \Psi(F \star K_{\varepsilon}) \; \cong \; \Psi(F)[\vec{\varepsilon}]$ for $\varepsilon$ such that and if $I$ is open then....
\end{lem}
\begin{pf}
 Using theorem~\ref{T:KSdecomposition} and the compatibility of convolution with direct sums and shifts, it is enough to prove the result for $\kk_I$ for an interval $I$.
\end{pf}
COMME POUR lE LEMMME 3.23 (il faut me rajouter là), il faut mettre les bonnes condtons sur $\varepsilon$ et expliquer les changements de degrés

\begin{cor}\label{C:MVcircPsiisId}
 There is a natural equivalence of functors 
 $(\overline{~\cdot~ })^{\text{MV}} \circ \Psi\simeq \text{id}_{\D^b_{\R c}(\kk_\R)}$ between the composition of $\Psi$ followed by the Mayer-Vietoris sheafification and the identity functor of $\D^b_{\R c}(\kk_\R)$.
\end{cor}
\begin{pf}
 Since both functors $(\overline{~\cdot~ })^{\text{MV}}$ and $ \Psi $ commutes with shifts and direct sums (propositions~\ref{P:PropertiesofMVSheafification} and \ref{P:PropertiesofPsi}), in view of the structure theorem~\ref{T:KSdecomposition}, it is enough to construct the equivalence on sheaves of the form $\kk_{I}$. Now proposition~\ref{P:PropertiesofMVSheafification}.3 (for $\varepsilon=0$) and Proposition~\ref{P:PsionIntervals} precisely give the result for an interval.
\end{pf}
\begin{cor}\label{C:fullorfiathful}
The functor  $\overline{(-)}^{MV}: \text{M-V}(\R)^{\s} \to  \D^b_{\R c}(\kk_\R)$ is essentially surjective and is a full functor. 

The functor $ \Psi:\D^b_{\R c}(\kk_\R)\to  \text{M-V}(\R)^{\s}$  is a faithful functor. 
\end{cor}
\begin{pf}

For  any    constructible sheaf $F^\bullet$, $\Psi(F^\bullet)$ is a 
strongly pointwise finite dimensional Mayer-Vietoris system and Corollary~\ref{C:MVcircPsiisId} gives a natural isomorphism $F^\bullet \cong \overline{(\Psi(F^\bullet))}^{MV} $. Therefore, $F^\bullet$ is in the essential image of $\overline{(-)}^{MV}$.  

\smallskip

Similarly the natural isomorphism  $(\overline{~\cdot~ })^{\text{MV}} \circ \Psi \simeq \text{id}_{\D^b_{\R c}(\kk_\R)}$ implies that $(\overline{~\cdot~ })^{\text{MV}}$ is surjective on morphisms and  $\Psi$ injective on morphisms which concludes the proof.
\end{pf}


\begin{cor}\label{C:MVdisO}
 Let $M\in \mbox{MV}(\R)^\s$ be a strongly pointwise finite dimensional Mayer-Vietoris system. Then 
 $$d_I^{\text{MV}}(M, \Psi ( \overline{M}^{MV})) = 0. $$
\end{cor}
In other words, though $\Psi \circ \overline{(\cdot)}^{MV}$ is not an equivalence, it maps an object to an object which is at distance $0$ fom itself.
\begin{pf}
 By statement 5 of Proposition~\ref{P:PropertiesofMVSheafification}, it is sufficient to prove that $\overline{M}^{MV}$ and $\overline{\Psi(\overline{M}^{MV})}^{MV}$ are isomorphic in $\D^b_{\R c}(\kk_\R)$. But Corollary~\ref{C:MVcircPsiisId} implies  $\overline{\Psi(\overline{M}^{MV})}^{MV} \cong \overline{M}^{MV}$ and the result follows.
\end{pf}



\begin{thm} One has equalities $$d_I(M, N) = d_C^(\overline{M}^{MV}, \overline{N}^{MV}), $$
$$ d_C(F,G) =d_I(\Psi(F), \Psi(G)).$$
\end{thm}


\subsection{\og $d_I^\Z = d_B$ \fg ~for complete level-set persistence}

\subsection{\og $d_I^\Z = \max_\Z d_I$\fg~}
On peut calculer la distance entre deux MV system comme le max des distances degrés par degrés




\section{Preliminaries}

\subsection{Notations}

We equip $\R^2$ with the product order $\leq$ and see the associated poset as a category. For any topological space $X$, we will denote by $\Op_X$ the category of its open subsets. Except when otherwise stated, $\R$ will be equipped of the euclidean topology. 

Let $\Delta^+ = \{(x,y)\in \R^2 \mid  x + y > 0 \}$, observe that it is a full sub-poset of $(\R^2,\leq)$. We identify $\Delta^+$ as the poset of open intervals of $\R$ through the poset morphism $$\iota : \Delta^+ \ni (x,y) \mapsto ]-x,y[ := \{\alpha \in \R \mid -x < \alpha < y\} \in \Op_\R $$

We fix $\kk$ a field, and will denote by $\Mod (\kk)$ the category of $\kk$-vector spaces, $\text{mod}(\kk)$ its full subcategory in which objects are finite dimensional vector spaces. For $P$ a poset, we define the category of persistent modules over $P$ as the functor category $\text{Fun}(P,\Mod(\kk))$, and its full subcategory of point-wise finite dimensional (pfd) persistent modules $\text{Fun}(P,\text{mod}(\kk))$. We denote by $\text{PSh(X)}$ the category of pre-sheaves of $\kk$-vector spaces over $X$ and $\Mod(\kk_\R)$ the category of sheaves of $\kk$-vector spaces over $\R$.  

Now observe that 

\subsection{Structure theorems for sheaves and level-set persistence}

structure and isometry theorems

\subsection{A refinement of the structure theorem for level set persistence}




\begin{prop}\label{P:BarMViso}
The functor $\overline{(-)}^{MV}: \text{MV-}(\R) \to \D^b_{\R c}(\R)$ is $1$-lipschitz: for any MV-systems $S$, $T$, one has
$$d_C(\overline{S}^{MV},\overline{T}^{MV}) \; \leq \; d_I^{MV}( S, T).$$
\end{prop}
\begin{pf}
Since the distance are given by the infimum of interleavings, it is enough to prove that if $S \sim_{\varepsilon}^{MV} T$, then $\overline{S}^{MV} \simeq_{\varepsilon} \overline{T}^{MV}$. 
Assume given $f=(f_i): S\to T[\vec{\varepsilon}] $ and $g=(g_i): T\to S[\vec{\varepsilon}]$ an $\varepsilon$-interleaving.
Then by definition~\ref{D:InterleavingforMV}, for all $i\in \Z$, the maps $f_i$ and $g_i$  give us that    $S_i \sim_{\varepsilon} T_i$. By Lemma~\ref{L:Barpreserveentre}, we obtain that 
$\overline{S_i} \sim_{\varepsilon} \overline{T_i}$; we write $\psi_i: \overline{S_i}\to \overline{T_i}*K_\varepsilon$ and $\kappa_i: \overline{T_i} \to \overline{S_i} \star K_\varepsilon$. Taking the direct sum of the (shifts of the) maps, and using that convolution commutes with shifts and direct sum,  we obtain maps
$$ \Psi: \overline{S}^{MV} =  \ell\Big(\bigoplus_{i\in \Z} \overline{S_i})[i]\Big)\stackrel{\ell(\bigoplus \psi_i [i])}\longrightarrow \ell\Big(\bigoplus_{i\in \Z} \overline{T_i})\star K_\varepsilon[i]\Big) 
\; \cong \; \ell\Big(\bigoplus_{i\in \Z} \overline{T_i})[i]\Big) \star K_\varepsilon 
\cong \overline{T}^{MV}\star K_\varepsilon$$ and similarly
$$\kappa: \overline{T}^{MV} =  \ell\Big(\bigoplus_{i\in \Z} \overline{T_i})[i]\Big)\stackrel{\ell(\bigoplus \kappa_i [i])}\longrightarrow \ell\Big(\bigoplus_{i\in \Z} \overline{S_i})\star K_\varepsilon[i]\Big) 
\; \cong \; \ell\Big(\bigoplus_{i\in \Z} \overline{S_i})[i]\Big) \star K_\varepsilon 
\cong \overline{S}^{MV}\star K_\varepsilon$$ which gives an $\varepsilon$-interleaving of sheaves.
\end{pf}

\begin{thm}
Let $X$ be a topological space, $f: X \to \R$ a continuous pfd map, and $(M_i^f)_{i\in \Z_{\geq 0}}$ its associated level-set persistence modules. Then, for $i\in\Z_\geq 0$ there is a one to one mapping between the death-quadrants appearing in the barcode of $M_i^f$ and the birth-quadrants appearing in the barcode of $M_{i+1}^f$ that does not intersect with the line of equation $x + y = 0$. 
\end{thm}

\begin{pf}
The proof relies on the fact that the family $(M_i^f)$ satisfies a stronger form of exactness, induced by the Mayer-Vietoris sequence. Indeed, we have for any \textcolor{red}{square} $(a,b,c,d)$, the long-exact Mayer-Vietoris sequence in $\Pe(\Delta^+)$ : 

$$...\stackrel{}{\longrightarrow} M_i^f[a] \stackrel{\alpha_i}{\longrightarrow} M_i^f[b] \oplus M_i^f[d] \stackrel{\beta_i}{\longrightarrow} M_i^f[c] \stackrel{\gamma_i}{\longrightarrow} M_{i-1}^f[a] \longrightarrow ...  $$

Now observe that since $\alpha_i$  and $\beta_i$ are linear combinations of the internal smoothing morphisms of $M_i^f$, they preserve any barcode decomposition of $M_i^f$.
\end{pf}

\bibliographystyle{alpha}
\bibliography{biblio}
\end{document}